\documentclass[10pt,pdf,hyperref={unicode}, dvipsnames]{beamer}
\documentclass[10pt,pdf,hyperref={unicode}, dvipsnames]{beamer}
%!TEX root = ../presentation.tex
\usepackage[english,russian]{babel}
% \usepackage[T2A,T1]{fontenc}
\usepackage[utf8]{inputenc}
% \usepackage{tikz}
\usepackage[unicode]{hyperref}
\usepackage{animate}
% \usepackage{pgfplots,standalone}
\usepackage{caption}
\usepackage[normalem]{ulem}
\usepackage
	{
		% Дополнения Американского математического общества (AMS)
		amssymb,
		amsfonts,
		amsmath,
		amsthm,
		physics
		}
% \usepackage{lmodern}
% \pgfplotsset{compat=newest} 
% \usetikzlibrary{%
%     decorations.pathreplacing,%
%     decorations.pathmorphing,%
%     patterns,%
%     angles,%
%     quotes,%
%     calc, %
%     3d, %
%     backgrounds, %
%     positioning%
% }


% Стиль презентации

\definecolor{darkcerulean}{rgb}{0.03, 0.27, 0.49}
 \usetheme{default}
 \usefonttheme{professionalfonts}
 %\usefonttheme{default}
 \usecolortheme{}
 % \usecolortheme{whale}
% \let\oldframe\enumerate
% \renewcommand{\frame}{%
% \oldframe
% \let\olditemize\itemize
% \renewcommand\itemize{\olditemize\addtolength{\itemsep}{100pt}}%
% }
 

% \setbeamercolor{frametitle right}{fg=white,bg=Brown!85}
% \setbeamercolor{frametitle}{fg=white,bg=Brown!85}
%\setbeamercolor{frametitle right}{fg=white,bg=black!85} %
\setbeamercolor{frametitle}{fg=white,bg=darkcerulean} % Цвет титульника
\setbeamercolor{item projected}{fg=white,bg=darkcerulean} % Цвет титульника
%\setbeamercolor{item projected}{fg=white,bg=black!85} % Цвет титульника
\setbeamertemplate{blocks}[rounded][shadow=true] %стиль блоков
\setbeamertemplate{itemize item}{\color{darkcerulean}$\bullet$}
\setbeamertemplate{headline}{}
\setbeamertemplate{footline}{} 
\setbeamertemplate{navigation symbols}{} % минус навигация
% \let\Tiny=\tiny % решает проблему со шрифтами в TexLive
%\setbeamertemplate
%	{footline}{
%		\color{black!40!white}
%		\quad\hfill
%		\insertframenumber/\inserttotalframenumber
%		\hfill\vspace{1cm}\quad
%	} 


\beamersetrightmargin{1cm} 
\beamersetleftmargin{1cm}

\setbeamertemplate{enumerate item}
{
	\usebeamercolor[bg]{item projected}
	\raisebox{1pt}{\colorbox{darkcerulean}{\color{fg}\footnotesize\insertenumlabel}}%
}

% \setbeamertemplate{itemize item}{%
% 	\usebeamercolor[bg]{item projected}%
% 	\raisebox{3pt}{{\colorbox{black!85}\footnotesize$\bf$\bullet}}%
% }

\setbeamercolor{item projected}{bg=darkcerulean,fg=white}
\setbeamercolor{title}{bg=darkcerulean,fg=white}

%\setbeamercolor{title}{bg=black!85,fg=white}

\setbeamertemplate{frametitle}
{	
	\nointerlineskip
	\begin{beamercolorbox}[sep=15pt,ht=1.9em,wd=\paperwidth]{frametitle}
		% \vspace{}%
		\strut\insertframetitle\strut
		\vskip-1.8ex%	
	\end{beamercolorbox}
}


\usepackage{mathtools}
\mathtoolsset{showonlyrefs=true} 

\usepackage{import}
\usepackage{pdfpages}
\usepackage{transparent}
\usepackage{xcolor}
\usepackage{calc}

\newcommand{\executeiffilenewer}[3]{%
    \ifnum\pdfstrcmp{\pdffilemoddate{#1}}%
    {\pdffilemoddate{#2}}>0%
    {\immediate\write18{#3}}\fi%
}
\newcommand{\includesvg}[1]{%
    \executeiffilenewer{#1.svg}{#1.pdf}%
    {inkscape -z -D --file=#1.svg %
    --export-pdf=#1.pdf --export-latex}%
    \import{./fig/}{#1.pdf_tex}
}

\newcommand{\mean}[1]{\langle#1\rangle}
\usepackage{caption}
\usepackage{subcaption}
\renewcommand{\phi}{\varphi}
%\usepackage[e]{esvect}
%\usepackage{animate}
%\renewcommand{\vec}{\vv}
\newcommand{\tM}{\widetilde{M}}
\begin{document}
\title[Моделирование морской поверхности]{Численное моделирование морской поверхности}

\author{Понур К.А.}

\institute{Национальный исследовательский Нижегородский государственный университет имени Н. И. Лобачевского \\ Радиофизический факультет}

\input{sections/presentation_titlepage}

% \section{Введение}
% \subsection{Цели работы}
\begin{frame}[t]
	\frametitle{Введение}
	\vfill
	\textbf{Цели: }\\
		% \vfill
		\begin{enumerate}
			% \item \sout{Получить зачёт по УНЭ.}
			\item Изучить принципы моделирования морской поверхности.

			\item Оптимизировать существующие алгоритмы.


		\end{enumerate}
		\vfill
% \end{frame}
% \subsection{Актуальность работы}
% \begin{frame}[t]

	\textbf{Актуальность работы: }

	\begin{enumerate}
		% \item Проведение испытаний оборудования до его изготовления
		\item Тестирование и разработка алгоритмов восстановления океанографической информации
		\item Оценка возможностей новых радиолокаторов
		\item Постановка численных экспериментов, в частности накопление статистических данных
	\end{enumerate}
	\vfill
\end{frame}

% \begin{frame}[t]\frametitle{Основные понятия}

% Для статистически однородного и стационарного поля $\zeta$ высот морского волнения справедливо следующее выражение для его корреляционной функции:

% \begin{equation}
%  M(\rho) = \langle{\zeta(r)\zeta(r+\rho)}\rangle.
% \end{equation}

% Она связана с спектром высот $S(k)$ морской поверхности:
% \begin{equation}
%     	M(\rho)=\int\limits_0^{\infty} S(k)\cos(k \rho) \dd{k},
% \end{equation}    
% Спектр уклонов морской поверхности связан со спектром высот соотношением $S_{\theta}(k)=k^2 S(k)$.

% Корреляционная функция наклонов:
% \begin{equation}
% 	M_{\theta}(\rho)=\int\limits_0^{\infty} k^2 S(k)\cos(k \rho) \dd{k}
% \end{equation}
%     % Если поверхность представляем как $\zeta(r)= \sum\limits_{i=1}^N a_i\cos(k_i r+ \phi)$, то корреляционная функция модельного поля определяется выражением: 
%     % \begin{equation}
%     % 	\tM(\rho)=\sum\limits_0^{N} b_i \cos(k_i \rho), \quad b_i=\frac{a_i^2}{2},
%     % \end{equation}
%     % $k_i$ -- абсцисса спектральной компоненты
 
%     % $b_i$ -- ордината спектральной компоненты

% \end{frame}



\begin{frame}[t]

	\frametitle{Двумерная модель поверхностного волнения}
	Представим морскую поверхность в виде суммы синусоид с детерминированными амплитудами и случайными фазами:
\begin{equation}
	\zeta(\vec r, t)= \sum\limits_{n=1}^N \sum_{m=1}^M A_n(k_n)\cdot 
		\Phi_{k_nm}(\phi_m) \cos(\omega_n t + \vec k_n \vec r + \psi_{nm}),
\end{equation}
$\psi_{nm}$ -- случайная фаза, $A_n$ -- амплитуда $n$-ой гармоники.

Амплитуда, которая является мощностью на интервале $\Delta k_n$, вычисляется по спектру моделируемой поверхности
\begin{equation}
	A_n(k_n)=\sqrt{2 S(k_n) \Delta k_n}
\end{equation}

$\Phi_{nm}$ -- азимутальное распределение, вычисляемое следующим образом:
\begin{equation}
	\Phi_{nm}(k_n,\phi_m)=\sqrt{\Phi(k_n,\phi_m) \Delta \phi},
\end{equation}
$\Delta \phi$ -- шаг по углу.

\end{frame}
\begin{frame}[t]
    \frametitle{Модель заостренной поверхности}
    \begin{equation}
        \footnotesize
        \begin{cases}
            \label{eq:surface2dcwm}
            z(\vec r,t) = \sum\limits_{n=1}^{N} \sum\limits_{m=1}^{M}
            A_n(\kappa_n) \cdot
            F_m(\kappa_n,\phi_m) \cos \qty(\omega_n t + \vec \kappa_n \vec r_0 +
            \psi_{nm}),    \\
            x = x_0 - \sum\limits_{n=1}^{N} \sum\limits_{m=1}^{M}
            A_n(\kappa_n) \cdot
            F_m(\kappa_n,\phi_m) \cos\phi_m \sin\qty(\omega_n t + \vec \kappa_n \vec r_0 +
            \psi_{nm}),\\
            y = y_{0} - \sum\limits_{n=1}^{N} \sum\limits_{m=1}^{M}
            A_n(\kappa_n) \cdot
            F_m(\kappa_n,\phi_m) \sin \phi_m \sin \qty(\omega_n t + \vec \kappa_n \vec
            r_0 + \psi_{nm}),
        \end{cases}
    \end{equation}
    где $\vec \kappa$ -- двумерный волновой вектор,  
    $\vec r_0 = (x_0, y_0)$, $\vec r = (x, y)$

    \begin{figure}
        \centering
        \includegraphics[height=0.5\textheight]{fig/evolution}
    \end{figure}

\end{frame}

\begin{frame}[t]
	\frametitle{Модель заостренной поверхности}
    \begin{figure}[h]
        \begin{subfigure}{0.49\linewidth}
            \centering
            \includegraphics[width=\linewidth]{fig/water/pdf_cwm}
        \end{subfigure}
        \begin{subfigure}{0.49\linewidth}
            \centering
            \includegraphics[width=\linewidth]{fig/water/surface_cwm}
        \end{subfigure}
    \end{figure}    
    \begin{minipage}{0.45\linewidth}
        \footnotesize
        \begin{equation}
            \begin{gathered}
            \sigma^2_{\alpha \beta \gamma} =  \int\limits_{}
            \frac{\kappa_x^\alpha 
            \kappa_y^\beta}{\kappa^{\gamma}} S(\vec \kappa) \dd \vec \kappa,\\
            \sigma_n^2 = \int\limits_{}^{} \kappa^n S(\vec \kappa) \dd \vec \kappa 
            \end{gathered}
        \end{equation}
        \begin{equation}
            \label{eq:char}
            \Phi(\theta) = (1 - i \theta \sigma_1^2 + \theta^2 \Sigma_1)
            \exp(-\frac{1}{2} \theta^2 \sigma_0^2),
        \end{equation}
        где $\Sigma_1 = \sigma^4_{111} - \sigma_{201}^2 \sigma_{021}^2$.
    \end{minipage}
    \hfill
    \begin{minipage}{0.45\linewidth}
        \footnotesize
        \begin{equation}
            \mean{z} = - \sigma_1^2, \quad \mean{z^2} = \sigma_0^2 - 2 \Sigma_1
        \end{equation}
    \end{minipage}
\end{frame}

\begin{frame}[t]{Метод <<отбеливания>> спектра}
\begin{figure}[h!]
    \begin{subfigure}{0.49\linewidth}
        \includegraphics[width=\linewidth]{fig/fig1}
    \end{subfigure}
    \begin{subfigure}{0.49\linewidth}
        \includegraphics[width=\linewidth]{fig/fig2}
    \end{subfigure}
    \begin{subfigure}{0.49\linewidth}
        \includegraphics[width=\linewidth]{fig/fig3}
    \end{subfigure}
\end{figure}
    
\end{frame}

\begin{frame}[t]{Изображение поверхностей}
    \begin{figure}[h]
        \begin{subfigure}{0.49\linewidth}
            \centering
            \includegraphics[width=1\linewidth]{img/heights5.png}
        \end{subfigure}
        \begin{subfigure}{0.49\linewidth}
            \centering
            \includegraphics[width=1\linewidth]{example-image-a}
        \end{subfigure}
        \begin{subfigure}{0.49\linewidth}
            \centering
            \includegraphics[width=1\linewidth]{example-image-a}
        \end{subfigure}
        \begin{subfigure}{0.49\linewidth}
            \centering
            \includegraphics[width=1\linewidth]{example-image-a}
        \end{subfigure}
    \end{figure}    
\end{frame}
\begin{frame}[t]
	\frametitle{Увеличение производительности}
    \begin{figure}[h]
        \begin{subfigure}{0.49\linewidth}
            \centering
            \includegraphics[width=\linewidth]{fig/water/gpucpu.pdf}
        \end{subfigure}
        \begin{subfigure}{0.49\linewidth}
            \centering
            \includegraphics[width=\linewidth]{fig/water/gpucpu1.pdf}
        \end{subfigure}
    \end{figure}
\end{frame}


\begin{frame}[t]
	\frametitle{Моделирование отраженного импульса}
    \begin{figure}[h]
        \centering
        \includegraphics[width=\linewidth]{fig/flat_wave1.pdf}
    \end{figure}
\end{frame}

\begin{frame}[t]
	\frametitle{Моделирование отраженного импульса}
    \begin{equation}
        \label{eq:brown}
        P(t) = A e^{-v} (1 + \erf(u)), \text{ где}
    \end{equation}
    \begin{gather}
        A = A_0 \exp{\frac{- 4}{\gamma} \sin^2 \xi},~
        u = \frac{t - \alpha \sigma_c^2}{\sqrt 2 \sigma_c},~
        v = \alpha\qty(t - \frac{\alpha}{2} \sigma_c^2)~
    \end{gather}
    в которых
    \begin{equation}
        \alpha = \delta - \frac{\beta^2}{4} = \frac{4}{\gamma}\cdot \frac{c}{h} \qty(\cos 2\xi - \frac{\sin^2 2\xi}{\gamma}),
    \end{equation}
    \begin{equation}
        \gamma = \frac{\ln 2}{2} \sin^2 \theta_{-3 dB},
        \sigma_c^2 =  \sigma_p^2 + \frac{\sigma^2}{c^2},
    \end{equation}
    $\xi \ll 1$ -- малое отклонение антенны от надира,  

    $\theta_{-3 dB}$ -- ширина
    диаграммы направленности антенны на уровне $-3dB$, 

    $h$ -- высота радиолокатора над поверхностью земли, 

    $c$ -- скорость света в вакууме, 

    $\sigma^2$ -- дисперсия высот взволнованной морской поверхности.


	%\hfill
	%\begin{minipage}{0.5\linewidth}
        %\begin{equation}
            %P(t) = A \exp{S_T(t - \frac{\tau}{2}) \qty(1 + \erf\frac{t -
        %\tau}{\sigma_L})}
        %\end{equation}
	%\end{minipage}
\end{frame}

\begin{frame}[t]
	\frametitle{Моделирование отраженного импульса}
\begin{equation}
    \label{eq:ice}
    P(t) = A \exp{ S_T (t - \frac{\tau}{2})} \qty(1 + \erf{\frac{t-
    \tau}{\sigma_L}}), \text{ где}
\end{equation}

 $S_T$ -- коэффициент наклона заднего фронта импульса, 

 $\tau$ -- эпоха,

 $\sigma_L$ -- ширина переднего фронта импульса, 

    \begin{figure}[h]
        \centering
        \def\svgwidth{0.8\linewidth}
        \includesvg{example_impulse1}
        \caption{Качественная форма импульса с обозначением основных параметров.}
        \label{fig:impuls}
    \end{figure}
\end{frame}
\begin{frame}[t]
	\frametitle{Моделирование отраженного импульса}
    \begin{figure}[h]
        \begin{subfigure}{\linewidth}
            \centering
            \def\svgwidth{0.8\linewidth}
            \includesvg{local_theta}
        \end{subfigure}
    \end{figure}
\end{frame}
\begin{frame}[t]
	\frametitle{Моделирование отраженного импульса}
    \begin{figure}[h]
        \begin{subfigure}{0.60\linewidth}
            \centering
            \includegraphics[width=\linewidth]{img/theta0}
        \end{subfigure}
        \begin{subfigure}{0.39\linewidth}
            \centering
            \includegraphics[width=\linewidth]{fig/theta}
        \end{subfigure}
    \end{figure}
\end{frame}
% \begin{frame}[t]\frametitle{Модель поверхностного волнения}
    
% \begin{figure}[h!]
% \begin{minipage}[h]{0.45\linewidth}
% 	\centering
% 	\includegraphics[width=\linewidth]{img/water7.png}
% 	% \caption{Моделирование высот морского волнения. $N=256, ~ U_{10}=7$  }
% 	\label{fig:water7}
% \end{minipage}
% \hfill
% \begin{minipage}[h]{0.45\linewidth}
% 	\centering
% 	\includegraphics[width=\linewidth]{img/water10.png}
% 	% \caption{Моделирование высот морского волнения. $N=256, ~ U_{10}=10$ }
% 	\label{fig:water10}
% \end{minipage}
% \end{figure}

% \end{frame}
\begin{frame}
\frametitle{Алгоритм ретрекинга}
\vskip -3pt
\def\imp{fig/retracking}
\begin{figure}
    \centering
    \begin{subfigure}{0.49\linewidth}
    При $t > t_{max}$ 
    \begin{equation}
        P(t) = 2A\exp{S_T\qty(t - \frac{\tau}{2})}
    \end{equation}
        \centering
        \includegraphics[width=1\linewidth]{\imp/imp_5_1}
        \begin{equation}
            \footnotesize
            \ln P(t) = \ln 2A + S_T(t - \frac{\tau}{2}) = S_T t + const
        \end{equation}
    \end{subfigure}
    \hfill
    \begin{subfigure}{0.49\linewidth}
        При $t < t_{max}$ 
        \begin{equation}
            \dv{\erf\frac{t - \tau}{\sigma_L}}{t} \gg 
            \dv{\exp{S_T(t - \frac{\tau}{2})}}{x}
        \end{equation}

        \centering
        \includegraphics[width=1\linewidth]{\imp/imp_5_2}
        \begin{equation}
            \footnotesize
            P(t) \approx A\qty(1 + \erf\frac{t - \tau}{\sigma_L})
        \end{equation}
    \end{subfigure}

\end{figure}
\end{frame}

\begin{frame}
    %\begin{subfigure}{0.49\linewidth}
        %\centering
        %\includegraphics[width=1\linewidth]{\imp/imp_5_3}
    %\end{subfigure}
\frametitle{Ретрекинг модельных импульсов}
\vskip -3pt
\begin{figure}
    \centering
    \begin{subfigure}{0.42\linewidth}
        \centering
        \includegraphics[width=1\linewidth,page=1]{fig/retracking/model}
    \end{subfigure}
    \hfill
    \begin{subfigure}{0.42\linewidth}
        \centering
        \includegraphics[width=1\linewidth,page=2]{fig/retracking/model}
    \end{subfigure}
    \hfill
    \begin{subfigure}{0.42\linewidth}
        \centering
        \includegraphics[width=1\linewidth,page=3]{fig/retracking/model}
    \end{subfigure}
    \hfill
    \begin{subfigure}{0.42\linewidth}
        \centering
        \begin{tabular}{|c|c|c|c|}
            \hline
            $h$, м      & $0.83$ & $1.36$ & $5.14$\\
            $\tilde h,$ м & $0.65$ & $1.49$ & $4.9$\\
            \hline
        \end{tabular}

        \vspace{\baselineskip}

        $h$ -- высота, заложенная в моделировании

        $\tilde h$ -- высота, полученная нами
    \end{subfigure}

\end{figure}
\end{frame}

\begin{frame}
\frametitle{Ретрекинг импульсов с Jason-3}
\vskip -3pt
\begin{figure}[ht]
    \centering
    \begin{subfigure}{0.42\linewidth}
        \centering
        \includegraphics[width=\linewidth, page=1]{fig/retracking/real}
    \end{subfigure}
    \hfill
    \begin{subfigure}{0.42\linewidth}
        \centering
        \includegraphics[width=\linewidth, page=2]{fig/retracking/real}
    \end{subfigure}
    \hfill
    \begin{subfigure}{0.42\linewidth}
        \centering
        \includegraphics[width=\linewidth, page=3]{fig/retracking/real}
    \end{subfigure}
    \hfill
    \begin{subfigure}{0.42\linewidth}
        \centering
        \begin{tabular}{|c|c|c|c|}
            \hline
            $h$, м      & $0.937 $ & $0.699$ & $1.075$ \\
            $\tilde h,$ м & $0.931$ & $0.703$ & $1.081$ \\
            \hline
        \end{tabular}

        \vspace{\baselineskip}

        $h$ -- высота, полученная NASA

        $\tilde h$ -- высота, полученная нами
    \end{subfigure}
    %\caption{Форма отраженного импульса в зависимости от времени, полученного с
    %радиовысотомера космической миссии Jason-3.}
    \label{fig:impulse_jason}
\end{figure}
\end{frame}

\begin{frame}

\frametitle{Заключение}
\vskip -3pt
\begin{figure}
	%\animategraphics[autoplay,loop,width=0.8\linewidth]{15}{anim/water}{0}{360}
\end{figure}
\end{frame}

\end{document}

\renewcommand{\phi}{\varphi}
\renewcommand{\epsilon}{\varepsilon}
\renewcommand{\div}{\operatorname{div}}
\begin{document}
\title[Измерение плотности плазмы]{Численное моделирование морской поверхности}

\author{%
	Понур К.А. %
}

\institute{Национальный исследовательский Нижегородский государственный университет имени Н. И. Лобачевского, \\ Радиофизический факультет}

%!TEX root = ../plasma.tex
\begin{frame}[plain]
	
	\begin{center}
		\small{\insertinstitute}
		\vspace{1cm}
	\end{center}
		\begin{beamercolorbox}[sep=8pt,center]{title}
			\usebeamerfont{title}\inserttitle
		\end{beamercolorbox}
		\vspace{0.1cm}
	\begin{flushright}
		\normalsize \textbf{Работу выполнил:}\\
		\large
		\insertauthor \\
		\vspace{0.5cm}
		\normalsize{\textbf{Научный руководитель:}\\}
		\large{Караев В.Ю.}
		\vfill
	\end{flushright}

	\centering{\small{\today }}
\end{frame}

\section{Введение}
\subsection{Цели работы}
\begin{frame}[t]
	\frametitle{Цели работы}
	% \textbf{Цели}\\
		\vfill
		\begin{enumerate}
			% \item \sout{Получить зачёт по УНЭ.}
			\item Изучить принципы моделирования морской поверхности.

			\item Оптимизировать существующие алгоритмы.


		\end{enumerate}
		\vfill
\end{frame}
\subsection{Актуальность работы}
\begin{frame}[t]

	\frametitle{Актуальность работы}

\end{frame}


\begin{frame}
	\frametitle{Основные понятия}

\end{frame}

\begin{frame}[t]

	\frametitle{Двумерная модель}
    
\end{frame}


\begin{frame}[t]

	\frametitle{Реальные и модельные поля}
    
\end{frame}

\begin{frame}[t]
	
	\frametitle{<<Отбеливание>> спектра}
   
\end{frame}

\begin{frame}[t]

	\frametitle{Модель поверхностного волнения}
    
\end{frame}

\end{document}