% Тип документа
\documentclass[a4paper,14pt]{extarticle}

% Шрифты, кодировки, символьные таблицы, переносы
\usepackage[utf8x]{inputenc}
\usepackage[T2A]{fontenc}
\usepackage[russian]{babel}
%\usepackage{latex2doc}

% Это пакет -- хитрый пакет, он нужен но не нужен
\usepackage[mode=buildnew]{standalone}
\usepackage{setspace}

%\usepackage[subfigure]{tocloft}
\usepackage{tikz}
\usetikzlibrary{shapes,arrows,positioning,fit,backgrounds,calc}

\usepackage
    {
        % Дополнения Американского математического общества (AMS)
        amssymb,
        amsfonts,
        amsmath,
        amsthm,
        % Пакет для физических текстов
        physics,
        % misccorr,
        % 
        % Графики и рисунки
        %wrapfig,
        graphicx,
        subcaption,
        float,
        pgfplots,
        pgfplotstable,
        %caption,
        color,
        booktabs,
        geometry,
        % 
        % Таблицы, списки
        makecell,
        multirow,
        indentfirst,
        %
        % Интегралы и прочие обозначения
        ulem,
        esint,
        esdiff,
        % 
        % Колонтитулы
        fancyhdr,
    } 

\usepackage{mathtools}
\mathtoolsset{showonlyrefs=true} 

%\usepackage[svgnames]{xcolor}
\usepackage{hyperref}
 % Цвета для гиперссылок
\definecolor{linkcolor}{HTML}{000000} % цвет ссылок
\definecolor{urlcolor}{HTML}{799B03} % цвет гиперссылок
 
\hypersetup{linkcolor=linkcolor,urlcolor=urlcolor, colorlinks=true,
citecolor=linkcolor}
\hypersetup{pageanchor=false}
% Увеличенный межстрочный интервал, французские пробелы
\linespread{1.5} 
\frenchspacing 

\newcommand{\mean}[1]{\langle#1\rangle}
\newcommand\ct[1]{\text{\rmfamily\upshape #1}}
\newcommand*{\const}{\ct{const}}
\renewcommand{\phi}{\varphi}
\renewcommand{\epsilon}{\varepsilon}
%\renewcommand{\sigma}{\varsigma}

\usepackage{array}
\usepackage{pstool}

\geometry       
    {
        left            =   2cm,
        right           =   2cm,
        top             =   2.5cm,
        bottom          =   2.5cm,
        bindingoffset   =   0cm
    }

%----------------------------------
\usepackage{import}
\usepackage{pdfpages}
\usepackage{transparent}
\usepackage{xcolor}
\usepackage{calc}

\newcommand{\executeiffilenewer}[3]{%
    \ifnum\pdfstrcmp{\pdffilemoddate{#1}}%
    {\pdffilemoddate{#2}}>0%
    {\immediate\write18{#3}}\fi%
}
\newcommand{\includesvg}[1]{%
    \executeiffilenewer{#1.svg}{#1.pdf}%
    {inkscape -z -D --file=#1.svg %
    --export-pdf=#1.pdf --export-latex}%
    \import{./fig/}{#1.pdf_tex}
}


\renewcommand{\contentsname}{Оглавление}
\usepackage{tocloft}
\usepackage{secdot}
\sectiondot{subsection}

 \input{python}
\begin{document}

\newgeometry{left=30mm,right=15mm,top=15mm,bottom=25mm,bindingoffset=0cm,headheight=15pt}
\begin{titlepage}
\begin{spacing}{1}
	\fontsize{11pt}{11pt} \selectfont
	{\centering
	\linespread{1}
	\noindent{\textbf{ МИНИСТЕРСТВО НАУКИ И ВЫСШЕГО ОБРАЗОВАНИЯ РОССИЙСКОЙ} \\[0.2em]  \textbf{ ФЕДЕРАЦИИ}}\\[13pt]
	\begin{spacing}{1.5}
		{\fontsize{12pt}{12pt} \selectfont\bf  Федеральное государственное автономное образовательное учреждение \\[-0.27em]
		высшего образования \\[-0.3em]
		<<Национальный исследовательский \\[-0.72em] Нижегородский
		государственный университет им. Н.И. Лобачевского>>
		}\\[4.5pt] 
	\end{spacing}
	\fontsize{12pt}{12pt} \selectfont
	Радиофизический факультет\\[16pt]
	Направление 03.03.03 <<Радиофизика>>\\[5pt]
	Профиль <<Фундаментальная радиофизика>>\\
	\vspace{30pt}
	ВЫПУСКНАЯ КВАЛИФИКАЦИОННАЯ РАБОТА\\
	\vspace{34pt}
	{{
            ЧИСЛЕННОЕ МОДЕЛИРОВАНИЕ МОРСКОЙ ПОВЕРХНОСТИ
	}}\\
	\vspace{34pt}}\fontsize{12pt}{12pt} \selectfont
	\noindent <<К защите допущен>>:\\[15pt]
	зав. кафедрой общей физики,\\[0.4em]
	д.ф.-м.н., профессор\hfill \rule{2cm}{1pt} Бакунов М.И. \hphantom{a\!a}\\[15pt]
	%
	%
	Научный руководитель:\\[0.4em]
	д.ф.-м.н., н.с. ИПФ РАН\hfill \rule{2cm}{1pt} Караев В.Ю. \hphantom{aa\!a}\\[15pt]
	%
	%
	Рецензент:\\[0.4em]
	должность,\\[0.4em] к.ф.-м.н., н.с. ИКИ РАН\hfill \rule{2cm}{1pt}
    Садовский И.Н.\hphantom{\,\,}\\[30pt]
	%
	%
	Консультант по технике безопасности:\\[0.4em]
	ст. преп. кафедры акустики, к.ф.-м.н.\hfill \rule{2cm}{1pt} Дерябин М.С.\hphantom{aa,}\\[15pt]
	%
	%
	Студент 4-го курса \hfill \rule{2cm}{1pt} Понур К.А.
    \hphantom{\,\,\,aaa}\\[15pt]
	\vfill
	\centering
	Нижний Новгород\\[0.4em]
	2020 год
\end{spacing}

\end{titlepage}
\clearpage
\restoregeometry

\tableofcontents
\newpage


\section{Введение}%
\label{sec:vvedenie}

Моделирование морской поверхности является интересной научной задачей, которая
привлекает к себе внимание ученых на протяжении длительного времени.
Несмотря на значительный прогресс, остается много вопросов, которые требуют
дальнейших исследований. Кратко рассмотрим основные используемые подходы и
подробно обсудим способ моделирования, который будет использоваться в данной
работе.  Для описания поверхностного волнения применяют уравнения гидродинамики
и в общем виде задача пока не по силам современной вычислительной технике.
Благодаря упрощениям и предположениям задача становится <<счетной>>, но всё
равно требует много вычислительных ресурсов, поэтому этот подход используется для
решения научно-исследовательских задач. 

В настоящее время активно применяются радиофизические методы дистанционного
зондирования для
решения прикладных задач, например, для измерения скорости, ветра, высоты
значительного волнения, температуры воды, диагностики разливов нефти и др. 
Несмотря  на значительные успехи существующая измерительная аппаратура не
всегда позволяет получить достаточно полное представление о состоянии
приповерхностного слоя океана, поэтому постоянно разрабатываются новые
радиолокационные системы.

Разработка новой измерительной аппаратуры дистанционного зондирования для
космических носителей является сложной научно-технической задачей решение
которой требует проведения полного комплекса исследований, включающего
теоретический анализ, численное моделирование и эксперимент. В ходе решения
ищется оптимальная схема измерения, анализируются требования к измерительной
аппаратуре и оценивается эффективность алгоритмов обработки. Проведение
численного эксперимента является необходимым этапом, поскольку  позволяет максимально быстро проанализировать разные варианты и предложить схему измерения и состав измерительного комплекса для этапа экспериментальных работ.

В данной работе, кроме моделирования морской поверхности, будет проведен 
численный эксперимент с орбитальным радиолокатором в целях не только проверить
качество моделируемой поверхности, но и оценить точность алгоритмов
восстановления данных, которые обычно используются при обработки данных
радиовысотомера.  

Мы последовательно рассмотрим этапы такого численного эксперимента, начиная  с
моделирования морского волнения и заканчивая тестированием алгоритмов
восстановления параметров волнения по импульсу, отраженного от моделируемой поверхности.


\newpage
%!TEX root = ../diplom.tex
\newcommand{\tK}{\widetilde K}
\section{Численное моделирование морского волнения}

Для выполнения вычислений в режиме реального времени применяются
мощные компьютеры. Это позволяет прогнозировать развитие морского волнения и
результаты активно используют океанологи и метеорологи. Однако за это
приходится «платить» низким пространственным разрешением, упрощением исходных
уравнений и моделированием только длинно-волновой составляющей спектра
волнения.  Для оценки эффективности работы радиолокационной аппаратуры больше
подходит хорошо известный подход, опирающийся на модель спектра волнения. В
этом случае морская поверхность представляется в виде набора синусоид
(гармоник), амплитуда которых вычисляется по спектру
волнения \cite{longe-higgins}, \cite{karaev}. При
таком подходе смоделированная морская поверхность утрачивает ряд свойств,
присущих реальной морской поверхности, но становится более удобной для счета и
моделирование может быть проведено на домашнем компьютере. Именно
этот подход выбран для моделирования морской поверхности в данном исследовании.
Надо отметить, что для выбранного подхода качество моделирования зависит от
используемого спектра волнения и от численной реализации процедуры
моделирования.

Предполагается, что гармоники не взаимодействуют друг с другом, поэтому возвышения поверхности, орбитальные
скорости, уклоны и другие характеристики волнения являются суммой
независимых гармоник.

\subsection{Общие сведения}%
\label{sec:obshchie_poniatiia}
Определим ряд общих понятий, описывающих возвышение взволнованной морской поверхности в рамках теории случайных пространственно-временных полей. В этом случае поверхность представляется в виде суммы синусоидальных волн со случайными фазами 
\begin{equation}
    \label{eq:surface}
    \xi(\vec r,t) = \sum\limits_{n=-\infty}^{\infty} 
        A_n(\vec \kappa_n) e^{i(\omega_n t + \vec \kappa_n \vec r + \psi_n)},
\end{equation}
где $\psi_n$ -- случайная фаза,
равномерно распределенная в интервале от $0$ до  $2 \pi$, 
$A_n (\vec \kappa_n)$ -- комплексная амплитуда гармоники с волновым числом
$\vec \kappa_n$ и временной частотой  $\omega_n$, связанной с  $\vec \kappa_n$ известным
дисперсионным соотношением
\begin{equation}
    \omega(\kappa) = \sqrt{\kappa g + \alpha \kappa^3},
\end{equation}
где $g$ -- ускорение свободного падения,  $\alpha$ -- коэффициент, 
зависящий от свойств жидкости.


Корреляционную функцию $K_{\xi}(\vec r,t)$ поля  $\xi(\vec r, t) $ определим
стандартным образом \cite{tihonov}
 \begin{equation}
    \label{eq:corr}
    K_{\xi}\qty[\vec r_1, \vec r_2, t_1,t_2] = \mean{\xi(\vec r_1,t_1)\xi^*(\vec r_2,t_2)}
\end{equation}

Поле высот в нашей задаче считаем стационарным в широком смысле, то есть 
$K_{\xi}\qty[\vec r_{1},\vec r_{2},t_{1},t_{2}] = K_{\xi}\qty[\vec \rho = \vec
r_{2} - \vec r_1, \tau=t_{2}-t_{1}]$. Будем считать, гармоники
независимыми друг от друга, а значит перекрестные члены в уравнении
\eqref{eq:corr} равны нулю.

Тогда корреляционная функция поверхности
\eqref{eq:surface} примет следующий вид
\begin{equation}
    \label{eq:surface_corr}
    K_{\xi}\qty[\vec \rho,\tau] = \sum\limits_{n=-\infty}^{\infty} 
    \frac{A_n^2}{2} 
    \exp{i \qty(\vec \kappa_n \vec \rho + \omega \tau)}
\end{equation}

Для решения задачи моделирования отраженного от морской поверхности импульса
достаточно рассматривать мгновенный снимок моделируемой поверхности, в момент
отражения
а значит можно положить $\tau = \const = 0$  и  тогда $K_\xi[\rho,\tau] = K_\xi [\rho]$.

В этом случае справедлива формула Винера-Хинчина \cite{tihonov}
\begin{equation}
    \begin{gathered}
    \label{eq:Viner-Hinchin}
        S_\xi(\vec k) \int\limits_{-\infty}^{\infty} K_\xi \qty[\vec \rho] \exp{- i
        \vec \kappa \vec \rho} \dd \rho \\
        K_\xi[\rho] = \frac{1}{2 \pi} \int\limits_{-\infty}^{\infty} S_\xi (\vec k) \exp{+ i
        \vec \kappa \vec \rho} \dd \rho. 
    \end{gathered}
\end{equation}


Предположим, что спектр морского волнения можно представить в виде функции с
разделяющимися переменными, где $S_{\xi}(\kappa)$ определяет зависимость
спектральной плотности мощности от волнового числа, а функция $\Phi(\kappa, \phi)$ -- 
описывает зависимость спектральной плотности мощности от азимутального угла для
выбранного волнового числа
\begin{equation}
    S_\xi(\vec \kappa) = S_\xi(\kappa) \Phi_\xi(\kappa, \phi),
\end{equation}
где $\kappa = \sqrt{\kappa_x^2 + \kappa_y^2}$,  $\phi = \arctg
\frac{\kappa_x}{\kappa_y}$. Для
удобства, угловое распределение нормируется так, чтобы
\begin{equation}
\int\limits_{-\infty}^{\infty} \Phi_\xi(\kappa,\phi) \dd \phi = 1.
\end{equation}

\subsection{Двумерная модель поверхностного волнения}%
\label{sec:dvumernaia_model_poverkhnostnogo_volneniia}

В соответствии с предыдущим разделом, для моделирования случайной поверхности
$\xi(\vec r,t)$ будем использовать её представление в виде суперпозиции
плоских волн с различными частотами и случайными фазами $\psi_{nm}$, бегущих
под разными азимутальными углами $\phi_m$ \cite{karaev}:
\begin{figure}[H]
    \centering
    \includegraphics[scale=1]{fig/image65}
    \caption{Плотность вероятности случайной фазы $\phi$.}
    \label{fig:phase}
\end{figure}

\begin{equation}
    \label{eq:surface2d}
    \xi(\vec r,t) = \sum\limits_{n=1}^{N} \sum\limits_{m=1}^{M}
    A_n(\kappa_n) \cdot
    F_m(\kappa_n,\phi_m) \cos \qty(\omega_n t + \vec \kappa \vec r + \psi_{nm}),
\end{equation}
где $\psi_{nm}$ -- случайная фаза, равномерно распределенная в интервале от $0$
до $2 \pi$ (см. рис. \ref{fig:phase}), $F_m(\kappa_n, \phi_m)$ -- азимутальное
распределение для гармоники с волновым числом  $\kappa_n$,  $\vec \kappa_n =
(\kappa_{nx}, \kappa_{ny})$ -- 
волновой вектор. 
%В соответствии с
%центральной предельной теоремой \cite{tihonov}. 

Амплитуда $n$-ой гармоники $A_n$ есть
мощность на интервале $\Delta \kappa_n$, которая вычисляется по спектру моделируемой
поверхности $S_\xi(\kappa)$. Пользуясь определением корреляционной
функции \eqref{eq:surface_corr} и формулой Винера-Хинчина
\eqref{eq:Viner-Hinchin} получим точное выражение для нахождения амплитуды
$n$-ой гармоники  $A_n$

\begin{gather}
    K_{\xi}[\vec \rho, \tau] = \frac{1}{(2 \pi)^2}  \int\limits_{-\infty}^{\infty}  S_{\xi}(\vec \kappa) e^{i \vec \kappa \vec \rho} \dd \vec k = 
    \frac{1}{(2 \pi)^2} 
        \int\limits_{-\infty}^{\infty}
        \int\limits_{- \pi}^{\pi} 
    S_\xi(\kappa) \Phi_\xi(\phi) \kappa e^{i \vec \kappa\vec \rho} \dd \kappa \dd \phi = \\
    = \frac{1}{(2 \pi)^2} \int\limits_{-\infty}^{\infty} \kappa S_\xi
    (\kappa) e^{i \vec \kappa 
    \vec \rho} \dd \kappa = \sum\limits_{n=-\infty}^{\infty} \frac{(A_n(\vec
\kappa_n))^2}{2} e^{i \vec \kappa_n \vec \rho} 
\end{gather}

\begin{equation}
    \label{eq:Amplitude}
    A_n(\kappa_n) = \frac{1}{2 \pi} \sqrt{\int\limits_{\Delta \kappa_n} 2
        \kappa S_\xi(\kappa)
    \dd \kappa}
\end{equation}

%При достаточно большом $n \to \infty$ ($\Delta \kappa_n \to 0$) можно интегрировать
%прямоугольником
%\begin{equation}
    %A_n(\kappa_n) = \frac{1}{2 \pi} \sqrt{ 2 \kappa S_\xi(\kappa_n) \Delta
    %\kappa_n}
%\end{equation}
%c погрешностью, пропорциональной $\Delta A_n \sim  \sqrt{\frac{\dd \kappa
    %S_\xi(\kappa)}{\dd \kappa}
%\Delta \kappa_n^2}$. 

Для удобства, введем новое обозначение для спектра
$S(\kappa_n)\equiv \kappa_n S_\xi (\kappa_n)$.

Аналогично вычислению амплитуд, можно вычислить азимутальное распределение $F_m$  следующим образом:
\begin{equation}
    F_{nm}(\kappa_n,\phi_m) = \sqrt{\int\limits_{\Delta \phi_m}
    \Phi_{\xi}(\kappa_n,\phi) \dd \phi},
\end{equation}
где $\Delta \phi = \frac{2\pi}{M}$ -- шаг по азимутальному углу.

\begin{figure}[ht]
    \def\spec{fig/water/spectrum}
    \begin{subfigure}{0.5\linewidth}
        \centering
        \includegraphics[scale=1,page=1]{\spec}
        \caption{}
    \end{subfigure}
    \begin{subfigure}{0.5\linewidth}
        \centering
        \includegraphics[scale=1,page=2]{\spec}
        \caption{}
    \end{subfigure}
    \caption{Спектр высот $S(\kappa)$ для разных скоростей ветра: синяя кривая
    - 5 м/с, красная кривая - 10 м/с, коричневая кривая - 15 м/с, (a)
Ku-диапазон, (b) C-диапазон.}
    \label{fig:spectrum_heights}
\end{figure}

\begin{figure}[ht]
    \def\angdistrib{fig/water/angles_distribution}
    \begin{subfigure}{0.5\linewidth}
        \centering
        \includegraphics[scale=1,page=1]{\angdistrib}
        \caption{}
    \end{subfigure}
    \begin{subfigure}{0.5\linewidth}
        \centering
        \includegraphics[scale=1,page=2]{\angdistrib}
        \caption{}
    \end{subfigure}
    \caption{Спектр $\Phi(\kappa, \phi)$ для разных соотношений  $\kappa /
    \kappa_m$}
    \label{fig:angles_distrib}
\end{figure}


Графики $S(\kappa)$ и  $\Phi_\xi(\kappa)$ для приведены на рис.
\ref{fig:spectrum_heights} и рис. \ref{fig:angles_distrib} соответственно
\cite{ryabkova}. 
Вычисления на рис. \ref{fig:spectrum_heights} выполнены для скоростей ветра 
$U_{10} = 5$ м/с (синяя кривая), 10 м/с (красная кривая) и 15 м/с (коричневая
кривая), также на  рис. \ref{fig:spectrum_heights} учитывается граничное волновое число
для моделирования поверхности для двух диапазонов излучения: $Ku$ и $C$. В
рамках двух масштабной модели  рассеивающей поверхности \cite{bass-and-fuks}.
Волновое число $\kappa_m$ соответствует  максимуму спектра волнения  $S(\kappa)$. Стоит
заметить, что с ростом скорости ветра число используемых гармоник, необходимых
для получения одинакового качества моделирования,
возрастает. 

Это обусловлено тем, что растет интервал волновых чисел $\kappa$, на котором
определен спектр волнения. 

На рис. \ref{fig:water} изображены поверхности,
На рис. \ref{fig:water_photo} представлена фотография взволнованной морской
поверхности для сравнения с рис. \ref{fig:water},
построенные по формуле \eqref{eq:surface2d}.

Для рис. \ref{fig:water}a доминантная длина волны равна 23 м, высота
значительного волнения 0.83 м, для рис. \ref{fig:water}b доминантная длина
волны равна 206 м, а высота значительного волнения 6.58 м.

\begin{figure}[h!]
    \begin{subfigure}{0.5\linewidth}
        \centering
        \includegraphics[width=\linewidth]{img/heights5}
        \caption{}
    \end{subfigure}
    \begin{subfigure}{0.5\linewidth}
        \centering
        \includegraphics[width=\linewidth]{img/heights15}
        \caption{}
    \end{subfigure}
    \caption{ Полутоновое изображение смоделированного поля высот для
        направления ветра $30^\circ$ и разных скоростей ветра
        (a) $U_{10} = 5 \text{м}/\text{c}$;
        (b) $U_{10} = 15 \text{м}/\text{c}$;
}
    \label{fig:water}
\end{figure}
\begin{figure}[h!]
    \centering
    \includegraphics[width=0.49\linewidth]{img/water_photo.png}
    \caption{Фотография водной поверхности}
    \label{fig:water_photo}
\end{figure}

Такой подход к моделированию морской поверхности является одним из самых простых и достаточно эффективным, но у него есть существенные недостатки.

Прежде всего, моделируемая поверхность получается симметричной, хотя реальная поверхность асимметрична: передний склон волны более крутой и короткий по сравнению с задним склоном.

Кроме того, площадь гребней меньше площади впадин для морского волнения, что
также не находит отражения в свойствах моделируемой поверхности. Эти отличия
модельной поверхности от морской поверхности не позволят смоделировать так
называемые поправки на состояние морской поверхности \cite{fu},
\cite{pustovoytenko}. 
Правильно смоделировать именно поправки на состояние крайне важно для получения
достоверных результатов при моделировании формы импульса
отраженного радиолокационного
сигнала. Как решить эту проблему обсудим в дальнейшем.


Для моделирования морской поверхности необходимо определиться с числом
гармоник. Надо отметить, что с ростом скорости ветра число используемых
гармоник, необходимых для получения одинакового качества моделирования, будет
возрастать. Это обусловлено тем, что увеличивается интервал волновых чисел
$\kappa$, на котором определен спектр волнения (см. рис.
\ref{fig:spectrum_heights}). 

Следующая задача, которую надо решить, связана с тем, как расположить гармоники
по оси волновых чисел. Максимуму спектра волнения соответствует волновое числа
$\kappa_m$, левую границу спектра определим как $\frac{\kappa}{4}$, а правую
границу обозначим $\kappa_{cut}$. Это значение будет различаться для Ku- и
C-диапазонов. Формула для определения была получена в \cite{karaev} и приведена  
в приложении \ref{sec:code}\ref{lst:spectrum}.


Самый простой вариант расположения гармоник это равномерный шаг, который можно определить следующим образом:
\begin{equation}
    \Delta \kappa = \frac{k_{cut}}{(N-1)}, 
\end{equation}
где $\kappa_{cut}$ -- граничное волновое число, $N$ -- число грамоник.



Критерием качества моделирования, а также оптимального выбора числа гармоник
была выбрана близость следующих корреляционных функций высот:
\begin{equation}
    \begin{gathered}
        \label{eq:KK}
        K[\rho] = \int\limits_{-\infty}^{\infty} S(\kappa) \cos(\kappa\rho) \dd \kappa\\
        \tK(\rho) = \sum\limits_{n=1}^{N} \frac{A_n^2}{2} \cos(\kappa_n \rho)
    \end{gathered}
\end{equation}




Сравнение корреляционной функции $\tK[\rho]$ полученной по модели, с
теоретической корреляционной функцией $K[\rho]$   позволит оценить качество
модели.  

Если посмотреть на форму спектра, то задача усложняется тем,
что спектр высот является узким и в основном сосредоточен вблизи пика
(длинноволновой составляющей спектра волнения).  Кроме того, равномерный шаг
приводит к появлению <<артефактов>>, что хорошо видно на рис.
\ref{fig:corr_h_lin}.
Вычисления выполнены дл полностью развитого ветрового волнения и трех скоростей
ветра: 5 м/с, 10 м/с и 15 м/с. Число гармоник для всех скоростей ветра было
выбрано равным 256. Для удобства сравнения разных скоростей ветра, при
построении, корреляционные функции были нормированы на дисперсию высот.
\def\correlation{fig/water/correlation}
\begin{figure}[H]
    \centering
    \begin{subfigure}{0.49\linewidth}
        \centering
        \includegraphics[scale=1,page=1]{\correlation}
    \end{subfigure}
    \hfill
    \begin{subfigure}{0.49\linewidth}
        \centering
        \includegraphics[scale=1,page=2]{\correlation}
    \end{subfigure}
    \begin{subfigure}{0.49\linewidth}
        \centering
        \includegraphics[scale=1,page=3]{\correlation}
    \end{subfigure}
    \caption{Нормированная корреляционная функция высот для равномерного распределения и 3-х
    скоростей ветра: (a) 5 м/с,  (b) 10 м/с, (c) 15 м/с и числе
гармоник $N=256$}
    \label{fig:corr_h_lin}
\end{figure}
Частично от артефактов можно избавиться, выбрав неравномерный шаг. Нужно задать
распределение таким образом, чтобы вблизи малых значений волнового числа
(вблизи пика спектра) плотность расположения гармоник была существенно выше,
чем при больших $\kappa$.

Можно предложить несколько вариантов неравномерного распределения и ниже
протестируем два варианта.

В качестве первого распределения выберем следующую формулу
\begin{equation}
    k_i = k_{min} + \frac{k_{cut} - k_{min}}{(N-1)^2} (i-1)^2
\end{equation}

Для второго распределения выберем <<логарифмический>> шаг и положения гармоник
определим следующим образом
\begin{equation}
    k_i = k_{i-1} \cdot \Delta  \kappa
\end{equation}

На рис. \ref{fig:corr_h_quad} и \ref{fig:corr_h_log} видно, что лучше всех
<<работает>> логарифмическое разбиение интервала волновых чисел.
\begin{figure}[H]
    \centering
    \begin{subfigure}{0.49\linewidth}
        \centering
        \includegraphics[scale=1,page=9]{\correlation}
    \end{subfigure}
    \hfill
    \begin{subfigure}{0.49\linewidth}
        \centering
        \includegraphics[scale=1,page=10]{\correlation}
    \end{subfigure}
    \begin{subfigure}{0.49\linewidth}
        \centering
        \includegraphics[scale=1,page=11]{\correlation}
    \end{subfigure}
    \caption{Нормированная корреляционная функция высот для логарифмического распределения и
    4-х скоростей ветра: (a) 5 м/с,  (b) 10 м/с, (c) 15 м/с и числе}
    \label{fig:corr_h_quad}
\end{figure}


\begin{figure}[ht]
    \centering
    \begin{subfigure}{0.49\linewidth}
        \centering
        \includegraphics[scale=1,page=5]{\correlation}
    \end{subfigure}
    \hfill
    \begin{subfigure}{0.49\linewidth}
        \centering
        \includegraphics[scale=1,page=6]{\correlation}
    \end{subfigure}
    \begin{subfigure}{0.49\linewidth}
        \centering
        \includegraphics[scale=1,page=7]{\correlation}
    \end{subfigure}
    \caption{Нормированная корреляционная функция высот для неравномерного распределения и
    3-х скоростей ветра: (a) 5 м/с,  (b) 10 м/с, (c) 15 м/с и числе}
    \label{fig:corr_h_log}
\end{figure}

Как было отмечено выше, с увеличением скорости ветра число гармоник, необходимых для получения
одинакового качества моделирования, возрастает. На их рис.\ref{fig:corr_h_log}
видно, что с увеличением скорости ветра отклонение <<модельной>> корреляционной
функции $\tK[\rho]$  от <<истинной>> $K[\rho]$   увеличивается и, чтобы
уменьшить ошибку моделирования, необходимо увеличить число гармоник (синусоид),
а это будет замедлять скорость моделирования морской поверхности.

Как показало тестовое моделирование, для получения <<качественной>> численной
реализации рассеивающей поверхности требуется большой число гармоник, что делает процесс вычислений
длительным.  Для уменьшения числа гармоник был рассмотрен следующий подход.

\subsection{Метод <<отбеливания>> спектра}
\label{subsec:metod_otbelivaniia_spektra_}
\subsubsection{Метод <<отбеливания>> спектра для одной переменной}%

Для оптимизации времени построения поверхности и уменьшения количества гармоник
без уменьшения качества моделирования, предлагается использовать следующий
метод.

Предположим, что при больших $\rho$ гармонические составляющие корреляционной
функции не зависят друг от друга и мы можем пренебречь их взаимной корреляцией.
Тогда мощность <<шума>> функции $\tK (\rho)$ определяется выражением
$\displaystyle \sigma^2_{\text{шум}} = \sum\limits_{n=1}^{N} \frac{1}{2}
\qty ( \frac{A^2_i}{2} )^2 \equiv \sum\limits_{n=1}^{N} \frac{b_i^2}{2}$.

В областях малых $\rho$, напротив, гармоники должны сильно взаимодействовать и
соответствующая мощность равна  $\tK^2(0) =
\qty(\sum\limits_{n=1}^{N} b_i)^2$ (см. \eqref{eq:KK} ).
Образуем величину
\begin{equation}
    \label{eq:Q}
    Q = \frac{\sigma_{\text{шум}}^2}{\tK^2(0)},
\end{equation}
которая характеризует относительную мощность шумов. Минимум этой величины
находится путём решения системы уравнений
\begin{gather}
    \frac{\partial Q}{\partial b_i} = 0, \text{ для } i=1,2,\dots, N. \\
    \frac{b_i \qty( \sum\limits_{n=1}^{N} b_i )^2 - 2 \sum\limits_{n=1}^{N} b_i
    \sum\limits_{n=1}^{N}  \frac{b_i^2}{2}}{\qty(\sum\limits_{n=1}^{N}
b_i)^4}=0
\end{gather}

Частным результатом её решения является $b_1 = b_2 = \dots = b_N$.

Спектр модельного поля при этом имеет близкий к белому вид, а выравнивание
амплитуд спектральных компонент поля $S(\kappa)$ сводится к разбиению области
определения спектра $[\kappa_{min},\kappa_{max}]$ на участки $\Delta
\kappa_i$, интегралы по
которым от функции  $S(\kappa)$ имеют одно и тоже значение $b_i = b_{0} =
\frac{\sigma^2}{N}$.

Заметим теперь, что рассуждая о способах разбиения интервала частот
$[\kappa_{min},
\kappa_{max}]$ на участки $\Delta \kappa_i$ мы оставляли нерешенным вопрос о выборе
расположения гармоник $\kappa_i$ внутри этих участков. Обычно  $\kappa_i$ ставится у
правой границы ячейки  $\Delta \kappa_i$. При этом, однако, оказывается, что
модельная корреляционная функция плохо совпадает с экспериментальной
корреляционной функцией в области малых  $\rho$. Для достижения лучшего
согласия следует потребовать сопряжения всех производных (от первого до $N$-го
порядка) функций $\tK[\rho]$ и  $K[\rho]$ при  $\rho=0$. 
Поскольку $K'[\rho] = \frac{\partial^2 K[\rho]}{\partial \rho^2}$, это условие эквивалентно
требованию сопряжения моментов спектра модельного и реального полей, которое
записывается в виде
 \begin{equation}
    \sum\limits_{i=1}^{N} b_i \kappa_i^{2p} 
    = \int\limits_{0}^{\infty} \kappa^{2p}S(\kappa) \dd \kappa, 
\end{equation}

Полученная система $N$ уравнений для $N$ неизвестных $\kappa_i$ не имеет общего
решения и потому может анализироваться лишь численно. Чтобы упростить решение
нашей задачи, потребуем облегченного, по сравнению с предыдущим, условия
сопряжения вторых моментов модельного и реального спектров высот
 \begin{equation}
    b_i \kappa_i^2 = \int\limits_{\Delta \kappa_i} \kappa^2 S(\kappa) \dd \kappa,
\end{equation}
где $b_i= A_i^2 / 2$

Из него непосредственно следует правило нахождения узлов $\kappa_i$ 
\begin{equation}
    \label{eq:ki}
    {
        \kappa_i = \sqrt{\frac{N}{\int\limits_{-\infty}^{\infty} S(\kappa) \dd
        \kappa} \int\limits_{\Delta k_i} \kappa^2
    S(\kappa) \dd \kappa}. 
    }
\end{equation}
Такой способ выбора расположения гармоник, как нетрудно убедиться, обеспечивает
сопряжение корреляционных функций реального и модельного полей по второй
производной в нуле, или, иначе говоря, равенство дисперсий кривизн этих
полей.

Формула \eqref{eq:ki} выведена для спектра высот поверхностного волнения. Когда
возникает необходимость моделирования уклонов, то необходима сделать замену
переменной $S(\kappa) \to k^2 S(\kappa)$, чтобы получить формулу для нахождения правила
расположения гармоник для уклонов


\begin{equation}
    \label{eq:ki_slopes}
    {
        \kappa_i = \sqrt{\frac{N}{\int\limits_{-\infty}^{\infty} \kappa^2
        S(\kappa) \dd \kappa } \int\limits_{\Delta \kappa_i}
    \kappa^4 S(\kappa) \dd \kappa}. 
    }
\end{equation}

Смоделировать качественную морскую поверхность для поля высот легко и не
прибегая к дополнительным методам и используя логарифмическое разбиение
частотной области. Проблемы возникают при моделировании поля наклонов,
корреляционная функция которого быстро принимает шумовой характер.  
На рис. \ref{fig:nodes} и рис. \ref{fig:nodes1} приведено сравнение
корреляционных функций уклонов и высот для расположения гармоник по формуле по
формуле \eqref{eq:ki_slopes} (отбеливание спектра уклонов). Хорошо заметно, что
предложенный метод действительно уменьшает шум у корреляционной функции
уклонов, в результате метод <<отбеливания>> дает лучший результат из всех рассмотренных подходов. 

\begin{figure}[h!]
    \centering
    \includegraphics[width=0.6\linewidth]{fig/correlation_angles_wa.pdf}
    \caption{ Нормированная корреляционная функция уклонов для логарифмического расположения
    гармоник (зеленая кривая) и расположения по методу отбеливания спектра
(синяя кривая) для скорости ветра 10 м/с}
    \label{fig:nodes}
\end{figure}

\begin{figure}[h!]
    \centering
    \includegraphics[width=0.6\linewidth]{fig/correlation_height_wa.pdf}
    \caption{ Нормированная корреляционная функция высот для логарифмического расположения
    гармоник (зеленая кривая) и расположения по методу отбеливания спектра
(синяя кривая) для скорости ветра 10 м/с}
    \label{fig:nodes1}
\end{figure}




\begin{figure}[h!]
    \centering
    \begin{subfigure}{0.49\linewidth}
        \centering
        \includegraphics[width=\linewidth]{fig/correlation_height_height2.pdf}
        \caption{}
    \end{subfigure}
    \begin{subfigure}{0.49\linewidth}
        \centering
        \includegraphics[width=\linewidth]{fig/correlation_angles_height2.pdf}
        \caption{}
    \end{subfigure}
    \caption{ Нормированная корреляционные функции высот (a) и уклонов (b) при расположении гармоник
    по методу <<отбеливания>> спектра по формуле \eqref{eq:ki} }
    \label{fig:ki}
\end{figure}

\begin{figure}[h!]
    \centering
    \begin{subfigure}{0.49\linewidth}
        \centering
        \includegraphics[width=\linewidth]{fig/correlation_height_slopes2.pdf}
        \caption{}
    \end{subfigure}
    \begin{subfigure}{0.49\linewidth}
        \centering
        \includegraphics[width=\linewidth]{fig/correlation_angles_slopes2.pdf}
        \caption{}
    \end{subfigure}
    \caption{Нормированная корреляционные функции высот (a) и уклонов (b) при расположении гармоник
    по методу <<отбеливания>> спектра по формуле \eqref{eq:ki_slopes} }
    \label{fig:ki_slopes}
\end{figure}

Из рис. \ref{fig:ki} и \ref{fig:ki_slopes} видно, что определение положения
гармоник по методу отбеливания является эффективным только для той переменной,
которая использовалась в процедуре отбеливания. Для другой переменной результат
получается не слишком хорошим, что свидетельствует о необходимости
использования другого подхода при необходимости одновременного моделирования
поля высот и поля уклонов.

Для решения задачи рассеяния электромагнитного излучения морской поверхностью, 
необходимо моделировать поле высот (определяет форму импульса) и поле уклонов,
которое определяет условие обратного рассеяния падающего изулучения.

\subsubsection{Метод <<отбеливания>> спектра для двух переменных}%
Для такой задачи необходима рассмотреть другую функцию
относительных шумов $Q$, например
\begin{equation}
    \label{eq:Q_modif}
    Q = \frac{\qty(\sigma^{\text{н}}_{\text{шум}})^2}{(\tK^\text{н}(0))^2}+
        \frac{\qty(\sigma^{\text{в}}_{\text{шум}})^2}{(\tK^\text{в}(0))^2},
\end{equation}
где индексы <<н>> и <<в>> соответствуют наклонам и высотам. Учитывая то, что
оба слагаемых в уравнении \eqref{eq:Q_modif} вещественны и положительны, то 
экстремум функции $Q$ можно найти, зная экстремум каждого слагаемого по отдельности. 


Тогда, гармоники, определяющие минимум первого слагаемого описываются
формулой \eqref{eq:ki}, а минимум второго -- формулой \eqref{eq:ki_slopes}.  
На рис. \ref{fig:13} представлено расположение гармоник по методу
<<отбеливания>> спектра для высот и наклонов. Число гармоник было взято
небольшим (N=32), чтобы лучше было заметно распределение, получаемое при
применении метода. Поскольку получившиеся распределения имеют мало общих
корней, то сложно написать одну функцию распределения гармоник, которая будет
удовлетворять и минимуму шума высот и минимуму шума уклонов. Поэтому
предлагается в дальнейшим объединять обе функции распределения. Подобное
совмещенное расположение гармоник представлено на рис. \ref{fig:14}, где синим
цветом обозначены 128 гармоник, расположенные по формуле для отбеливания
спектра высот \eqref{eq:ki}, а красным цветом обозначены 128 гармоник, расположенных по формуле
для отбеливания спектра уклонов \eqref{eq:ki_slopes}.

\begin{figure}[H]
    \begin{subfigure}{0.49\linewidth}
        \includegraphics[width=\linewidth]{fig/fig1}
        \caption{}
    \end{subfigure}
    \begin{subfigure}{0.49\linewidth}
        \includegraphics[width=\linewidth]{fig/fig2}
        \caption{}
    \end{subfigure}
    \caption{Расположение гармоник для отбеливания (a) высот, (b) уклонов}
    \label{fig:13}
\end{figure}

\begin{figure}[H]
    \begin{subfigure}{0.49\linewidth}
        \includegraphics[width=\linewidth]{fig/fig3}
        \caption{Совмещенное расположение гармоник для отбеливания}
    \end{subfigure}
    \begin{subfigure}{0.49\linewidth}
        \includegraphics[width=\linewidth]{fig/water/whitening}
        \caption{Корреляционная функция наклонов для различного расположения
        гармоник в частотной области: (a) логарифмическое распределение, (b)
    метод <<отбеливания>> спектра}
    \end{subfigure}
    \caption{}
    \label{fig:14}
\end{figure}

Таким образом, двумерный вариант метода отбеливания является эффективным
способом выбора расположения гармоник для численного моделирования морской
поверхности, задаваемой моделью спектра.


\subsubsection{Аппаратное ускорение моделирования}%
\label{ssub:apparatnoe_uskorenie_modelirovaniia}


В прошлом разделе мы обсуждали как можно ускорить без потери качества процесс моделирования за счет уменьшения количества гармоник в спектре волнения.

Когда с математической точки зрения все оптимизировано можно перейти к
программной оптимизации: поскольку основное время моделирования приходится на
суммирование в цикле по формуле \eqref{eq:surface2d} , на который приходится
$N\times M \times X \times Y$ итераций, где $N$ – число гармоник в частотном
спектре, $M$ – число гармоник в азимутальном распределении, X – размер сетки
вдоль оси $x$, $Y$ – размер сетки вдоль оси $y$. Этот цикл требует больших
затрат мощности и времени и именно его мы можем значительно ускорить благодаря
его внутренней простоте.

Современные центральные процессоры (CPU) уже давно имеют в своём распоряжении
несколько (обычно 4-8) вычислительных ядер, которые в нередко в несколько
потоков могут производить вычисления.

Самое очевидное, что можно сделать – это выполнять програмный код не в одном потоке процессора, а специальным образом разделить координатную сетку на блоки и вычислять каждый блок в отдельном потоке.

Но ещё быстрее можно произвести вычисления на графическом процессоре (GPU).
Современные GPU имеют порядка 1000 вычислительных ядер, что позволяет очень
сильно ускорить процесс моделирования за счет распараллеливания вычислений
между ядрами.

На рисунке рис. \ref{fig:gpucpu} приведено сравнение многопоточных вычислений на CPU и GPU из
одной ценовой категории. Поверхность моделировалась при $N=2048$, $M=512$. При
вычислении поверхности $256\times256$ точек центральному процессору для этого
требуется 2030 секунд, в то время как графический процессор справится с этой
задачей лишь за 71 секунду.

\begin{figure}[H]
    \centering
    \begin{subfigure}{0.49\linewidth}
        \centering
        \includegraphics[]{fig/water/gpucpu.pdf}
    \end{subfigure}
    \begin{subfigure}{0.49\linewidth}
        \centering
        \includegraphics[]{fig/water/gpucpu1.pdf}
    \end{subfigure}
    \caption{Сравнение времени моделирования двумерной поверхности на CPU и
    GPU: (a) время вычисления поверхности на сетке размером $X \times Y$;
(b) относительная скорость вычислений $t_{GPU}/t_{CPU}$;}
    \label{fig:gpucpu}
\end{figure}

Программную реализацию вычислений на графическом процессоре можно посмотреть в
приложении \ref{sec:code} в листинге \ref{lst:kernel}.

\subsection{Заостренная морская поверхность}
\label{sub:cwm}

Как отмечалось ранее, при моделировании морской поверхности синусоидами мы
получаем нулевое среднее значение высот, что не позволяет смоделировать
поправки на состояние морской поверхности. 

Ниже предлагается модель поверхности у которой средний уровень не равен нулю.



\subsubsection{Двумерный случай}%
\label{ssub:odnomernyi_sluchai}

Рассмотрим для начала задачу моделирования двумерной поверхности суммой гармоник с детерменированными амплитудами и случайными фазами
 \begin{equation}
     z = \sum\limits_{j=0}^{N} A_j \cos(k_j x + \psi_j)
 \end{equation}

Чтобы получить модель заостренной волны введем нелинейное преобразование координат
\begin{equation}
    \qty{x,z(x)} \longrightarrow \qty{x + D(x),z(x)},
\end{equation}
где $D(x)$ горизонтальное смещение
\begin{equation}
    D(x) =  \frac{i}{2\pi} \int\limits_{-\infty}^{\infty}   \hat z e^{ikx} \dd{k},
\end{equation}
а $S(k)$ -- прямое Фурье преобразование исходной поверхности
\begin{equation}
    S(k) = \int\limits_{-\infty}^{\infty} z(x) e^{-ikx} \dd x 
\end{equation}

В нашем случае, функция $D(x)$ примет вид: 
\begin{equation}
    \begin{cases}
    x = x_{0} \underbrace{
    - \sum\limits_{j=0}^{N} A_j \sin(k_j x_0 + \psi_j)
    }_{D(x)} \\
    z = \sum\limits_{j=0}^{N} A_j \cos(k_j x_{0} + \psi_j)
    \end{cases}
\end{equation}

Иными словами мы будем моделировать волнение не суммой гармонических функций, а 
суммой трохоид. 

Для того, чтобы наше преобразование $D(x)$ имело физический смысл
необходимо, чтобы для каждой $j$-ой гармоники выполнялось соотношение 
\begin{equation}
    A_j k_j \ll 1 
\end{equation}

\paragraph{Статистические моменты}%

Запишем характеристическую функцию нового случайного процесса $z(x_0(x))$ по
определению
 \begin{equation}
    \label{eq:Phi1}
    \Theta(i\theta) = \mean{ e^{i \theta z(x_0(x))}}
\end{equation}
Поскольку процесс $z(x_0)$ стационарный, то от \eqref{eq:Phi1} можно перейти к
\begin{equation}
    \label{eq:Phi2}
    \Theta(i\theta) = \lim_{L \to \infty} \frac{1}{2L} \int\limits_{-L}^{L} e^{i \theta z(x_0)
    }\dd x = 
    \lim_{L \to \infty} \frac{1}{2L} \int\limits_{-L}^{L} e^{i \theta z(x_0)} \qty( 1 + D'(x_0) ) \dd x_0
\end{equation}

Поскольку $z(x_0)$ стационарный процесс, а  $D'(x_0)$ стационарен по нашему
определению, то  \eqref{eq:Phi2} преобразуется к виду
\begin{equation}
    \label{eq:Phi}
    \Theta(i\theta) = (1 - i \theta \sigma_1^2) 
    \exp(-\frac{1}{2} \theta^2 \sigma_0^2),
\end{equation}
где $\sigma^2_n = \int\limits_{-\infty}^{\infty}  k^n S(k) \dd k$ -- момент
$n$-го порядка спектра волнения.

Зная характеристическую функцию не сложно получить необходимые статистические
моменты дифференцируя \eqref{eq:Phi}
\begin{equation}
    m_n = i^{-n} \dv[n]{\Theta(i\theta)}{\theta} \eval_{\theta = 0}
\end{equation}

Следовательно, среднее и дисперсия случайного процесса $z(x_0)$ будут
равны
\begin{gather}
    \mean{z} = - \sigma_1^2, \quad \mean{z^2} = \sigma_0^2 \\
    \mean{z^2} - \mean{z}^2 = \sigma_0^2 - \sigma_1^4
\end{gather}

Также не сложно получить связь уклонов в смещенных координатах $x$ с наклонами
в несмещенных координатах $x_0$ пользуясь определением уклонов
 \begin{equation}
    z'(x) = \dv{z(x)}{x} = \frac{z'(x_0)}{1 + D'(x_0)}
\end{equation}



\subsubsection{Трехмерный случай}%

Для трехмерного случая Пирсон \cite{pierson} представил решение
линеаризованных уравнений движения для невязкой жидкости в лагранжевых
координатах. Он показал, что в глубокой воде положение частиц на свободной поверхности задается следующими параметрическими уравнениями
\begin{equation}
    \begin{cases}
        \label{eq:surface2dcwm}
        z(\vec r,t) = \sum\limits_{n=1}^{N} \sum\limits_{m=1}^{M}
        A_n(\kappa_n) \cdot
        F_m(\kappa_n,\phi_m) \cos \qty(\omega_n t + \vec \kappa_n \vec r_0 +
        \psi_{nm}),    \\
        x = x_0 - \sum\limits_{n=1}^{N} \sum\limits_{m=1}^{M}
        A_n(\kappa_n) \cdot
        F_m(\kappa_n,\phi_m) \cos\phi_m \sin\qty(\omega_n t + \vec \kappa_n \vec r_0 +
        \psi_{nm}),\\
        y = y_{0} - \sum\limits_{n=1}^{N} \sum\limits_{m=1}^{M}
        A_n(\kappa_n) \cdot
        F_m(\kappa_n,\phi_m) \sin \phi_m \sin \qty(\omega_n t + \vec \kappa_n \vec
        r_0 + \psi_{nm}),
    \end{cases}
\end{equation}
где $\vec \kappa$ -- двумерный волновой вектор,  
$\vec r_0 = (x_0, y_0)$, $\vec r = (x, y)$


\paragraph{Статистические моменты}
\label{par:statisticheskie_momenty}
В трехмерном случае вычисления аналогичны двумерному случаю, но более
громоздкие.  

Введем смешанный $\sigma_{\alpha \beta \gamma}^2$ и начальный $\sigma_n^2$ моменты спектра волнения
\begin{equation}
    \sigma^2_{\alpha \beta \gamma} =  \int\limits_{} \frac{\kappa_x^\alpha
    \kappa_y^\beta}{\kappa^{\gamma}} S(\vec \kappa) \dd \vec \kappa,\quad
    \sigma_n^2 = \int\limits_{}^{} \kappa^n S(\vec \kappa) \dd \vec \kappa 
\end{equation}
можно получить следующую характеристическую функцию для трехмерного волнения
\begin{equation}
    \label{eq:char}
    \Phi(\theta) = (1 - i \theta \sigma_1^2 + \theta^2 \Sigma_1)
    \exp(-\frac{1}{2} \theta^2 \sigma_0^2),
\end{equation}
где $\Sigma_1 = \sigma^4_{111} - \sigma_{201}^2 \sigma_{021}^2$.

Из этой характеристической функции можно получить необходимые моменты процесса
\begin{equation}
    \mean{z} = - \sigma_1^2, \quad \mean{z^2} = \sigma_0^2 - 2 \Sigma_1
\end{equation}
 
На рис. \ref{fig:pdf}b представлены срезы трехмерной морской поверхности 
для стандартного подхода и метода заостренной волны. На рис. \ref{fig:pdf}a
изображена теоретическая плотность вероятности наклонов для двух подходов. 

 \begin{figure}[h!]
    \centering
     \begin{subfigure}{0.49\linewidth}
        \centering
        \includegraphics[]{fig/water/pdf_cwm}
        \caption{}
    \end{subfigure}
    \hfill
     \begin{subfigure}{0.49\linewidth}
        \centering
        \includegraphics[]{fig/water/surface_cwm.pdf}
        \caption{}
    \end{subfigure}
    \caption{(a) Плотность вероятности уклонов $W_{\xi_x}(z)$ для линейной
        поверхности (синяя кривая) и заостренной поверхности (красная кривая) в
        зависимости от высот $z$ при скорости ветра 10 м/с \\
    (b) Срез поля высот морской поверхности для стандартного подхода (синяя
кривая) и модели заостренной поверхности (красная кривая) при скорости ветра 10
м/с. Пунктиром показан средний уровень соответствующей поверхности}
    \label{fig:pdf}
\end{figure}
На рис. \ref{fig:evolution} представлена эволюция во времени гребня волны для
двух методов. 

\begin{figure}[H]
    \centering
    \includegraphics[width=0.8\linewidth]{fig/evolution}
    \caption{Эволюция поверхности, построенной стандартным подходом в сравнении
    с моделью заостренной поверхности}
    \label{fig:evolution}
\end{figure}



На практике средний уровень морской поверхности не совпадает с тем, что может
определить альтиметр. Этот эффект возникает из-за того, что площадь впадин на
поверхности превышает площадь гребней, а значит во впадинах будет больше
отражающих зеркальных точек. 
Из вида характеристической функции $\Phi(\theta)$ в формуле  \eqref{eq:char} мы
можем найти связь плотности вероятности наклонов обычной поверхности с
заостренной
\begin{equation}
    \tilde P_{\xi_x}(z) = 
    P_{\xi_x}(z)\qty(1 + 
                    \frac{\Sigma_1}{\sigma_0^2} -
                    \frac{\sigma_1^2}{\sigma_0^2} z -
                    \frac{\Sigma_1}{\sigma_0^4}z^2), 
\end{equation}
где $P_\xi(z)$ -- гауссовая плотность вероятности наклонов линейной
поверхности,  $z$ -- высоты морской поверхности.


На рис. \ref{fig:pdf} изображен график функции $\tilde P_{\xi_x}$ в сравнении с
функцией $P_{\xi_x}(z)$. Можно заметить, что область нулевых наклонов, а значит
и отражающих точек функции $\tilde P_{\xi_x}(z)$ смещается в сторону
отрицательных высот, что приводит к изменению длительности импульса отраженного
от такой поверхности импульса. 
Это приводит к изменению формы переднего фронта импульса, излучаемого
радиолокатором. 
%О значении этого эффекта речь пойдет следующих разделах.


\newpage
%!TEX root = ../diplom.tex

\section{Моделирование процесса измерения}



При малых углах падения механизм обратного  рассеяния является квазизеркальным
и отражение происходит на участках волнового профиля, ориентированных
перпендикулярно падающему излучению. Тогда в формировании отраженного сигнала
будут участвовать только площадки, ориентированные нормально к излучению. 
Поэтому для моделирования рассеяния нам необходимо знать не только высоту в
точке, но и уравнение касательной к ней плоскости, для этого необходимо знать
наклоны $\zeta_x$ и  $\zeta_y$ в искомой точке
\footnote{Под словосочетанием <<отражающая точка>> подразумевается физически
бесконечно малая площадка, характерные размеры  которой больше длины волны
радиолокатора}. 

Зная координаты радиолокатора  $(x_{rad},y_{rad},H_0)$, координаты точки на
поверхности $(x,y,\zeta)$, можем из геометрии (см. рис. \ref{fig:local_theta})

\begin{equation}
    \label{eq:local_theta}
    \cos \theta = \frac
    {\zeta_x (x - x_{rad}) + \zeta_y (y - y_{rad}) - (\zeta - H_0) }
        {
           \sqrt{(x - x_{rad})^2 + (y - y_{rad})^2 + (\zeta - H_0)^2 }
           \sqrt{\zeta_x^2 + \zeta_y^2 + 1}
        }
\end{equation}

Вероятность того, что угол $\theta$ будет точно равен нулю и произойдет
зеркальное отражение для случайной выбранной точки очень мала, поэтому имеет
смысл рассматривать квазизеркальное отражение и вводить ограничение на
максимально допустимый локальный угол отражения. 

Нахождение всех зеркальных точек на характерном пятне радиолокатора  $> 1
\text{ км}^2$ представляет собой ресурсоемкую задачу. Но поскольку формирование
импульса носит статистический характер, то мы может выбирать гораздо меньшую
выборку зеркальных точек. 

Процесс создания такой выборки продемонстрирован на рис. \ref{fig:mirror:a}-\ref{fig:mirror:c}. 

Для смоделированной поверхности  рис. \ref{fig:mirror:a} для некоторой
выборки точек вычисляются по формуле \eqref{eq:local_theta} локальные углы
падения. Квазизеркальными позже считаются те, для которых угол меньше одного
градуса $\theta < 1^\circ$. Выборку можно делать несколькими способами,
например создать её выбирая случайные точки на координатной сетке или проходить
координатную сетку с равномерным шагом. 
Выборка на рис. \ref{fig:mirror:b} и \ref{fig:mirror:c} получена вторым
способом. 

\begin{figure}[h!]
    \centering
    \includesvg{local_theta}
    \caption{Геометрия определения локального угла падения. Красной линией
    обозначена касательная плоскость к рассматриваемой отражающей точке
$(x,y,\zeta)$}
    \label{fig:local_theta}
\end{figure}

\begin{figure}[h]
    \centering
    \begin{subfigure}{0.65\linewidth}
        \centering
        \includegraphics[width=\linewidth]{fig/impulse/fig1}
        \caption{Моделирование поверхности при скорости ветра $U=5$ м/с}
        \label{fig:mirror:a}
    \end{subfigure}
    \begin{subfigure}{.49\linewidth}
        \centering
        \includegraphics[width=\linewidth]{fig/impulse/fig2}
        \caption{Локальный угол отражения от поверхности для радиолокатора
        находящегося на высоте $H=1000$ км в точке с координатой (100, 100)}
        \label{fig:mirror:b}
    \end{subfigure}
    \begin{subfigure}{.49\linewidth}
        \centering
        \includegraphics[width=\linewidth]{fig/impulse/fig3}
        \caption{Положение зеркальных точек поверхности \ref{fig:mirror:a} для
        радиолокатора находящегося над точкой (100,100) }
        \label{fig:mirror:c}
    \end{subfigure}
    \label{fig:mirror}
    \caption{}
\end{figure}

Теперь, для вычисления поля вблизи приемной антенны радиолокатора нам
необходимо просуммировать отраженное от квазизеркальных точек поле
(см.рис. \ref{fig:mirror:c}).  


Как было сказано с предыдущих разделах, амплитуда поля излученного антенной
спадает по гиперболическому закону. Тогда амплитуда поля вблизи точки отражения
$(x,y,\zeta)$
будет определяться как (см. геометрию задачи на рис. \ref{fig:local_theta})
\begin{equation}
    \label{eq:}
    E_{sur} \sim \frac{E_0}{R_1} e^{-ikR_1} \cdot G(x,y,\theta_0), 
\end{equation}
Следовательно, вблизи приемной антенны амплитуду можно записать как
\begin{equation}
    \label{eq:E}
    E \sim \frac{E_{sur}}{R_1} e^{-ikR_1} \cdot G(x,y,\theta_0) =
    \frac{E_0}{R_1^2} e^{-2ikR_1} \cdot G^2(x,y,\theta_0), 
\end{equation}

Остается только проинтегрировать уравнение \eqref{eq:E} по всем отражающим
точкам 
\begin{equation}
    \label{eq:}
    E \sim \sum\limits_{i=1}^{M} \frac{E_0}{R_i^2} \exp{-2ikR_i}
    G^2(x,y,\theta_0)
\end{equation}
где $M$ -- количество точек,  $x_i,y_i$ -- координаты  $i-$ой отражающей точки,
 $R_i$-- расстояние от спутника до  $i-$ой точки.

\newpage
%!TEX root = ../diplom.tex
\subsection{Теоретический отраженный импульс}
\begin{figure}[h]
    \centering
    \def\svgwidth{0.8\linewidth}
    \includesvg{geometry}
    \caption{Геометрия задачи вычисления отклика радиолокатора на плоскую
    поверхность с учетом отклонения антенны от надира}
    \label{fig:geometry}
\end{figure}
Посчитаем теоретически отклик плоской морской поверхности $P_{FS}$ на сигнал с
радиолокатора.
Предполагаем известными диаграмму направленности $G(\theta)$, мощность
излученной энергии как функцию времени $P(t)$ и длину волны излучения
$\lambda$.

%Рассмотрим импульсный радиолокатор, который через равные или квазиравные
%промежутки времени подает  на антенну напряжение следующего вида
%\begin{equation}
    %\label{eq:U}
    %U(t) = 
    %\begin{cases}
        %\Re{\tilde U(t) e^{i \omega t}}, & \text{ при } t\geq 0 \\
        %0                         , & \text{ при } t<0.
    %\end{cases}
%\end{equation}

%Мощность такого сигнала будет равна
%\begin{equation}
    %P(t) = \frac{\tilde U(t)^2}{2}
%\end{equation}
%Поскольку радиолокатор излучает сферическую волну, её амплитуда будет
%уменьшаться по закону $\sim \frac{1}{r}$, 


Рассматривая малую плоскую площадку $m$, мы можем
составить уравнение, описывающее отраженную от неё мощность из следующих
четырех множителей:
\begin{equation}
    \label{eq:Pfsm}
    P_{FS_m} = 
    \frac
        {P\qty(t - \frac{2r_m}{c}) G_m}
        {4 \pi r_m^2} 
    \cdot \sigma_mA_m
    \cdot \frac{1}{4 \pi r_m^2} 
    \cdot \frac{G_m \lambda^2}{4 \pi}=
    \frac
        {P\qty(t - \frac{2r_m}{c})G_m^2 \lambda^2 \sigma_m}
        {(4 \pi)^3 r_{m}^4},
\end{equation} 
где 
$r_m$ -- расстояние от радара до рассеивающей площадки ,
$\sigma_m$ -- удельная эффективная площадь рассеяния площадки,
$G_m$ -- диаграмма направленности антенны в направлении на рассеивающую
площадку,
$A_m$ -- площадь площадки.

Первый множитель в уравнении  \eqref{eq:Pfsm}  соответствует плотности мощности
излучаемого сигнала. Второй множитель характеризует энергию падающего
излучения, которая излучается в направлении приемника, то есть эффективную
площадь рассеяния. Третий множитель
характеризует рассеяние в пространстве отраженной мощности из-за
сферичности волны. Четвертый
коэффициент это апертура антенны.

%Форма функции $P_{FS}$ ... 

%Из работы \cite{moore-and-williams} 

Для того, чтобы найти полную мощность переизлученного сигнала от интересующей
нас поверхности,
разобьем всю поверхность на элементарные площадки $\dd A$ и проинтегрируем
по ним 

\begin{equation}
    P_{FS}(t) = \frac{\lambda^2\mean{\sigma}}{(4 \pi)^3 } \int\limits_{} 
    \frac
        {P\qty(t - \frac{2r}{c}) G^2(r,\theta,\phi) }
        {r^4} 
    \dd A
\end{equation}

 Из геометрии задачи (cм. рис. \ref{fig:geometry}) задачи можно найти связь между
 азимутальным углом $\theta$, полярными углами  $\phi$,  $\tilde \phi$ и
 отклонением антенны от положения надира
 \begin{equation}
     \label{eq:cos:theta}
     \cos \theta = 
     \frac{\cos \xi + \frac{\rho}{h} \sin \xi \cos(\tilde \phi - \phi)}{\sqrt{1
     + (\frac{\rho}{h})^2}}
 \end{equation}
 Поскольку боковые лепестки по мощности гораздо меньше главного лепестка, то
 пренебрежем ими и положим диаграмму направленности   равной следующей функции
 \begin{equation}
     \label{eq:Gapprox}
     G(\theta) = G_0 e^{-\frac{2}{\gamma} \sin^2 \theta}
 \end{equation}

 Подставим \eqref{eq:cos:theta} в \eqref{eq:Gapprox}, учтем, что  элемент
 поверхности можно записать как $\dd A = \rho \dd \rho \dd \psi$ и тогда
 интеграл преобразуется к виду  (учтем, что \newline$r = \sqrt{h^{2} + \rho^{2}}$)
 \begin{multline}
     P_{FS}(t) = \frac{G_0^2 \lambda^2}{(4 \pi)^3 h^4}
     \int\limits_{0}^{\infty} \int\limits_{0}^{2 \pi}   
     \frac{P\qty(t - \frac{2h}{c}\sqrt{1+ \epsilon^2})}{(1+\epsilon^2)^2} \sigma^o(\psi)
     \\
     \cdot \exp{-\frac{4}{\gamma} \qty[1 - \frac{\cos^2 \xi}{1+ \epsilon^2}] + b
     + a\cos(\tilde \phi - \phi) - b \sin^2(\tilde \phi - \phi)} \dd \phi \rho
     \dd \rho,
 \end{multline}
 где 
 $\epsilon = \frac{\rho}{h}$,
 $a = \frac{4\epsilon}{\gamma} \frac{\sin 2 \xi}{(1+ \epsilon^2)}$,
 $b= \frac{4\epsilon^2}{\gamma} \frac{\sin^2 \xi}{(1+\epsilon^2)}$,
 в рамках наших задач, нас не будет интересовать абсолютное значение мощности,
 а только  вид её зависимости от времени. 

 Браун в своей работе \cite{brown} вычислил этот интеграл и показал, что он
 равен
 \begin{multline}
     P_{FS} = \frac{G_0^2 \lambda^2 c}{4(4 \pi)^2 L_p h^3} \cdot
     \frac{\sigma^o(\psi)}{(\frac{ct}{2h})^3} 
     \cdot \exp{
         -\frac{4}{\gamma} 
         \qty[
         \cos^2 \xi - \frac{\cos 2\xi}{(\frac{ct}{2h})^2}
     ]}
     \\
     \cdot (1+\epsilon^2)^2\sum\limits_{n=0}^{\infty} \frac{(-1)^n
     \Gamma(n+\frac{1}{2})}{\sqrt \pi \Gamma(n+1)}\qty[\qty(\frac{ct}{2h})^2 -1
     \tan\xi]^n \\
     \cdot I_n \qty(\frac{4}{\gamma} \sqrt{\frac{c \tau}{n}} \sin 2 \xi),
     \text{ при } t\geq 2h /c
 \end{multline}
 и $P_{FS} = 0 \text{ при } t< 2h/c$


 Это выражение можно упростить, переходя к новому времени $\tau = t - 2h / c$,
 где  $2h / c$ -- время задержки между излучением и приемом сигнала. Учитывая,
 что в масштабах спутниковой альтиметрии  $\frac{c \tau}{h} \ll 1$, получим 

 \begin{multline}
     \label{eq:Pfs:final}
     P_{FS}(\tau) = \frac{G_0^2 \lambda^2 c \sigma^o(\psi_0)}{4(4 \pi)^2 L_p h^3}
     \exp{
         -\frac{4}{\gamma}\sin^2 \xi - \frac{4c}{\gamma h} \tau \cos 2 \xi
     }  \\
     \cdot \sum\limits_{n=0}^{\infty} \frac{(-1)^n \Gamma(n+\frac{1}{2})}{\sqrt
     \pi \Gamma(n+1)} \qty[\sqrt{\frac{c\tau}{h}}\tan \xi]^n
 I_n\qty(\frac{4}{\gamma} \sqrt{\frac{c \tau}{h}} \sin 2 \xi)
     \text{ при } \tau\geq 0
 \end{multline}
 и $P_{FS} = 0, \text{ при } \tau < 0$


 Рассмотрим теперь отдельно сумму из уравнения \eqref{eq:Pfs:final}. Если
 переобозначить $Y=\frac{4}{\gamma} \sqrt{\frac{c\tau}{h}} \sin 2\xi$, то сумма
 примет вид
 \begin{equation}
     \label{eq:sum}
     I_0(Y) \cdot \qty{ 1 + \sum\limits_{n=1}^{\infty} 
     \frac
        {(-1)^n \Gamma(n+\frac{1}{2})}
        {\sqrt \pi \Gamma(n+1)} 
        \cdot \frac{I_n(Y)}{I_0(Y)} \qty[\frac{\gamma Y}{8 \cos^2 \xi}]^n
 }
 \end{equation}
 Поскольку $Y\ll 1$ и $\xi \ll 1$, то множитель  $\qty[\frac{\gamma Y}{8 \cos^2
 \xi}]^n $ будет быстро сходиться к нулю. Следовательно сумму $n$ слагаемых мы
 можем приближенно заменить лишь одним слагаемым при $n=0$.
 \begin{multline}
     \label{eq:Pfs:approx}
     P_{FS}(\tau) = \frac{G_0^2 \lambda^2 c \sigma^o(\psi_0)}{4(4 \pi)^2 L_p h^3}
     \exp{
         -\frac{4}{\gamma}\sin^2 \xi - \frac{4c}{\gamma h} \tau \cos 2 \xi
     }  \\
     \cdot  I_0\qty(\frac{4}{\gamma} \sqrt{\frac{c \tau}{h}} \sin 2 \xi)
     \text{ при } \tau\geq 0
 \end{multline}

Можно прибегнуть к ещё одному упрощению и разложить в ряд функцию Бесселя в
уравнении \eqref{eq:Pfs:approx}:
\begin{equation}
    I_0(\zeta) = \sum\limits_{n=0}^{\infty} 
    \qty(\frac{\zeta^2}{4})^n \cdot 
    \qty(\frac{1}{n!})^2, \text{ где } \zeta=\frac{4}{\gamma}\sqrt{\frac{c \tau}{h}
    }\sin 2 \xi
\end{equation}
поступим также, как Amarouche \cite{amarouche} в своей статье  и оставим только два первых члена
разложения, которые, в свою очередь, совпадают с разложением экспоненты
\begin{equation}
    I_0(\zeta) \approx 1 + \frac{\zeta^2}{4} = e^{\frac{\zeta^2}{4}}
\end{equation}
  
Тогда функция $P_{FS}$ примет вид
\begin{equation}
    \label{eq:Pfs}
    P_{FS}(\tau) = A \exp{-\frac{4}{\gamma} \sin^2 \xi} \exp{
        -\frac{4c}{\gamma h}
            \qty(\cos 2\xi - \sin^2 2\xi )\tau}
\end{equation}

 %Пропущу кусок
 %про 
 %Интегрируя почленно, получааем
 %\begin{multline}
     %P_{FS}(t) = 
     %\frac
        %{2 \sqrt \pi G_0 \lambda^2 \sigma^o(\psi_0)}
        %{(4 \pi)^3 L_p h^4}
     %\cdot\sum\limits_{n=0}^{\infty} 
     %\frac
     %{(-1)^n \Gamma(n+\frac{1}{2})}
     %{\Gamma(n+1)}\\
     %\cdot \int\limits_{0}^{\infty} \qty(\frac{2b}{a})^2 I_n(a) 
     %\cdot \exp{-\frac{4}{\gamma}\qty(1 - \frac{\cos^2 \xi}{1+
     %\epsilon^2})+b} 
     %\cdot \frac
     %{\delta\qty(t - \frac{2h}{c} \sqrt{1+ \epsilon^2})}
     %{(1+\epsilon^2)^2}
     %\rho \dd \rho,
 %\end{multline}
 %где $I_n(a)$ -- функция Бесселя, $\Gamma(n)$ -- гамма-функция Эйлера.



%Из геометрических соображений, можем определить что
%\begin{equation}
    %\label{eq:tanxi}
    %\tan \xi = \sqrt{\frac{c \tau}{h}}
%\end{equation}

Зная отклик плоской морской поверхности на сигнал с радиовысотомера мы
можем перейти к вычислению отклика на взволнованную морскую поверхность.
С точки зрения физики, различия будут в том, что теперь не вся
поверхность может отражать сигнал в нужном направлении, а только
зеркально ориентированные площадки на поверхности. С радиотехнической
точки зрения, взволнованная морская поверхность является линейным
фильтром с импульсной переходной характеристикой $q(t)$.  Тогда отклик
взволнованной поверхности можно вычислить выполняя свертку
\begin{equation}
    \label{eq:PRSconv}
    P_{RS}(t) = q(t) * P_{FS}(t) 
\end{equation}

Результирующая форму импульса будет сверткой функции отклика на плоскую
поверхность $P_{FS}$ и функции распределения зеркальных площадок $q$
 \begin{equation}
    P(\tau) = \int\limits_{-\infty}^{\infty} q\qty(\tau
    -  \tilde\tau) P_{FS}(\tilde \tau) \dd{\tilde \tau} 
\end{equation}

Стоит заметить, что из-за малого значения параметра $\xi$ и  масштаба задачи 
$(c \tau)/h\ll 1$ функция $P_{FS}$ изменяется гораздо медленнее функции плотности зеркальных точек $q$, а значит можно записать приближенное равенство \begin{equation}
    \label{eq:Psim}
    P \approx 
    \begin{cases}
        P_{FS}(0) \int\limits_{0}^{\infty} q\qty(\tau -
        \tilde \tau) \dd{\tilde \tau}, &\text{ при } \tau <0\\
        P_{FS}(\tau) \int\limits_{0}^{\infty} q\qty(\tau -
        \tilde \tau) \dd{\tilde \tau}, &\text{ при } \tau <0\\
    \end{cases}
\end{equation}

Поскольку корреляция между соседними площадками мала, а количество площадок
велико, то согласно
центральной предельной теореме \cite{central-limit-theorem} мы можем положить распределение $q$ гауссовым
\begin{equation}
    \label{eq:q}
     q\qty(\tau) = 
        \frac
        {1}
        {\sqrt{2 \pi \qty(\frac{2 \sigma_s^2}{c})}}
    \exp{-\frac{\tau^2}{2 \pi \qty(\frac{2 \sigma_s^2}{c})^2}}
\end{equation}

%Посчитав свертку, получаем, что
%\begin{equation}
    %P(\tau) \approx P_{FS}(\tau) \qty[ 1 + \erf\qty(\frac{c\tau}{2 \sqrt 2 \sigma_s})]
%\end{equation}



%Согласно статье \cite{cite:moore-and-williams} можно связать дисперсию $\sigma_p$ в \eqref{eq:PTR} c
%временным разрешением альтиметра $r_t$:  
%\begin{equation}
    %\label{eq:sigmap}
    %\sigma_p = \frac{1}{2 \sqrt{2 \ln 2}} r_t
%\end{equation}

%Согласно работе Брауна  \cite{brown}, мы можем выразить $FSSR$ 

Напоследок, следует учесть, что наш приемник (радиолокатор) тоже является
линейной системой с некоторой импульсной характеристикой $P_{T}(t)$. Поэтому необходимо к
уравнению \eqref{eq:PRSconv} добавить ещё одну свертку. 
В работе \cite{moore-and-williams}  показано, что отклик радиолокатора
имеет гауссовую форму
\begin{equation}
    \label{eq:Pt}
    P_T(\tau) \sim \exp{-\frac{\tau^2}{2 \sigma_p^2}},
\end{equation}
где $\sigma_p = \frac{1}{\sqrt  2\ln 2} T \approx 0.425 T$, $T$ --
временн\'{о}е
разрешение антенны, по порядку величины обычно равное $T\sim 3$ нс.

Тогда, результирующий импульс будет равен
\begin{equation}
    \label{eq:Pconv}
    P(t) = P_{FS}(t) * q(t) * P_T(t)
\end{equation}
В свертке \eqref{eq:Pconv}, c учетом \eqref{eq:Pfs}, \eqref{eq:Pt} и
\eqref{eq:q}, каждый множитель представляет собой экспоненту. Свертку от трех
экспонент не трудно посчитать аналитически. 

Получаем окончательную формулу для сигнала на приемнике радилокатора
\begin{equation}
    \label{eq:brown}
    P(t) = A e^{-v} (1 + \erf(u)), \text{ где}
\end{equation}
\begin{gather}
    A = A_0 \exp{\frac{- 4}{\gamma} \sin^2 \xi},\\
    u = \frac{t - \alpha \sigma_c^2}{\sqrt 2 \sigma_c},\\
    v = \alpha\qty(t - \frac{\alpha}{2} \sigma_c^2),\\
\end{gather}
в которых

\begin{equation}
    \alpha = \delta - \frac{\beta^2}{4} = \frac{4}{\gamma}\cdot \frac{c}{h} \qty(\cos 2\xi - \frac{\sin^2 2\xi}{\gamma}),
\end{equation}
\begin{equation}
    \gamma = \frac{\ln 2}{2} \sin^2 \theta_{-3 dB},
\end{equation}
\begin{equation}
    \sigma_c^2 =  \sigma_p^2 + \frac{\sigma_s^2}{c^2},
\end{equation}
$\xi \ll 1$ -- малое отклонение антенны от надира,  $\theta_{-3 dB}$ -- ширина
диаграммы направленности антенны на уровне $-3dB$, $h$ -- высота радиолокатора
над поверхностью земли, $c$ -- скорость света в вакууме, $\sigma_s^2$ -- 
дисперсия высот взволнованной морской поверхности.

График функции \eqref{eq:brown} изображен на рис. \ref{fig:impuls}.

\subsection{Восстановление параметров морской поверхности.}%
\label{sub:vosstanovlenie_parametrov_morskoi_poverkhnosti_}

Зная зависимость принятого сигнала от параметров взволнованной морской
поверхности, мы можем восстанавливать их по форме импульса. Это можно сделать,
аппроксимируя практический импульс теоретической формулой и извлекая из
получившегося графика необходимые коэффициенты.


Однако решать подобную задачу для формулы \eqref{eq:brown} довольно сложно
из-за сложной зависимости восстанавливаемых параметров и в  их большом
количестве.  Это может приводить к большим вычислительным ошибкам даже при большом соотношении сигнал-шум.

\begin{figure}[h]
    \centering
    \def\svgwidth{0.8\linewidth}
    \includesvg{example_impulse1}
    \caption{Качественная форма импульса с обозначением основных параметров.}
    \label{fig:impuls}
\end{figure}

Поэтому для решения задачи ретрекинга предлагается, использовать менее физичную, но более наглядную запись формулы
\eqref{eq:brown}
\begin{equation}
    \label{eq:ice}
    P(t) = A \exp{ S_T (t - \frac{\tau}{2})} \qty(1 + \erf{\frac{t-
    \tau}{\sigma_L}}), \text{ где}
\end{equation}

$S_T$ -- коэффициент наклона заднего фронта импульса, 
 $\tau$ -- эпоха
 $\sigma_L$ -- ширина переднего фронта импульса, 

\paragraph{Поиск наклона заднего фронта}%
\label{par:nakhozhdenie_s_t_}

Формула \eqref{eq:ice}, хороша тем, что можно найти некоторые коэффициенты, не
прибегая к сложным методам оптимизации. После прохождения пика импульса, функция ошибок
становится  медленно меняющейся функцией и можно записать приближенное  равенство
\begin{equation}
    \label{eq:t>tmax}
    P(t) = 2A \exp{S_T \qty(t - \frac{\tau}{2})}, \text{ при } t > t_{max},
\end{equation}
где $t_{max}$ -- ордината пика импульса.

Логарифмируя \eqref{eq:t>tmax}
\begin{equation}
    \ln P(t) = \ln 2A + S_T( t - \frac{\tau}{2}) = S_T t + \const 
\end{equation}
мы получаем линейную функцию времени. Значит, построив логарифм формы импульса при
$t>t_{max}$ и найдя коэффициент наклона получившейся прямой мы можем найти
наклон заднего фронта $S_T$. Подобная процедура проведена на рис.\ref{fig:S_T}
 \begin{figure}[h]
    \centering
    \includegraphics[width=0.75\linewidth]{fig/imp08_7_1.pdf}
    \caption{Поиск наклона заднего фронта}
    \label{fig:S_T}
\end{figure}

\paragraph{Поиск ширины переднего фронта}%
Как видно из рис.\ref{fig:erf}, при $t<t_{max}$ функция ошибок
$\erf\qty(\frac{t - \tau}{\sigma_L})$ ведет себя
быстрее экспоненты, а значит можно написать приближенное равенство
\begin{equation}
    \label{eq:erf}
    P(t) \approx A \qty(1 + \erf\frac{t - \tau}{\sigma_L})
\end{equation}

\begin{figure}[h]
    \centering
    \includegraphics[width=0.75\linewidth]{fig/imp08_7_2.pdf}
    \caption{Поиск ширины переднего фронта исходя из формы функции
    $\erf\qty(\frac{t-\tau}{\sigma_L})$}
    \label{fig:erf}
\end{figure}

Аппроксимируя импульс при $t<t_{max}$ формулой \eqref{eq:erf} мы получим оценку
коэффициентов $A,~\tau,~\sigma_L$. 
%При этом, интересовать нас на этом этапе
%будут в основном  эпоха $\tau$ и ширина переднего фронта  $\sigma_L$.


 Имея оценки параметров аппроксимации по различным
участкам функции $P(t)$ мы можем использовать формулу  \eqref{eq:ice} для всего
импульса 
\begin{equation}
    P(t) = A \exp{ S_T (t - \frac{\tau}{2})} \qty(1 + \erf{\frac{t-
    \tau}{\sigma_L}}).
\end{equation}
с начальными условиями для параметров $A, S_T, \tau, \sigma_L$, полученных на
предыдущих этапах. 



Ознакомиться с реализацией можно в приложении  \ref{sec:code} в листинге \ref{lst:retracking}. 




\paragraph{Восстановление параметров поверхности}%
Не сложно найти связь коэффициентов в формуле \eqref{eq:brown} и \eqref{eq:ice}:

\begin{equation}
    \label{eq:params}
    \begin{gathered}
        S_T = - \alpha,\\
        \sigma_L = \sqrt 2 \sigma_c,\\
        \sigma_c^2 = \sigma_p^2 + \qty(\frac{2\sigma_s}{c})^2.
    \end{gathered}
\end{equation}

Из соотношений \eqref{eq:params} восстанавливается значение дисперсии высот
(высоты значительного  волнения). 
Из амплитуды импульса мы можем узнать сечение обратного рассеяния, которое с
помощью различных регрессионных моделей моделей позволяет оценить скорость приводного ветра.

\subsection{Восстановление параметров модельных поверхностей}
\label{sub:retracking_model}

Чтобы оценить насколько точно работает выбранный нами алгоритм восстановления
параметров поверхности в реальных условиях, смоделируем морскую поверхность с
известными параметрами такими как: скорость приводного ветра, дисперсия высот,
дисперсия наклонов и применим алгоритм ретрекинга к этой поверхности. Так мы
сможем получить необходимые сведения о стабильности и точности алгоритма
восстановления и внести в него корректировки, если это будет необходимо.

В разделе \ref{sec:pulse_modeling} был описан процесс получения отраженного
импульса от известной модельной поверхности. 
Программная реализация
представлена в приложении \ref{lst:retracking}. 

Для получения импульса с соотношением сигнал/шум  таким же, как у качественного
трека с радиовысотомера требуется просуммировать отраженную мощность от
нескольких миллионов зеркальных точек, что требует длительного времени
вычислений. В разделе \ref{ssub:apparatnoe_uskorenie_modelirovaniia} был описан
способ более быстрых подсчетов благодаря вычислению на графическом процессоре,
что позволяет посчитать отраженный импульс за конечное время. 

На рис.\ref{fig:model_pulse_retracking} представлены отраженные от модельных
поверхностей импульсы при разных скоростях ветра. Для получения качественного
отраженного импульса потребовалось около $8\cdot 10^{6}$ зеркальных точек.

Применяя к импульсам на рис.\ref{fig:model_pulse_retracking} алгоритм
ретрекинга, получаем следующие восстановленные параметры:
\begin{align}
    \label{eq:retracking}
    & h = 0.83, \quad \tilde h = 0.65 \text{ для } U_{10} = 5~~\frac{\text{м}}{\text{с}}\\
    & h = 1.36, \quad \tilde h = 1.49 \text{ для } U_{10} = 10~\frac{\text{м}}{\text{с}}\\
    & h = 5.14, \quad \tilde h = 4.9  \text{ для } U_{10} = 15~\frac{\text{м}}{\text{с}}\\
\end{align}
где $h$ -- высота значительного волнения, известная из моделирования
поверхности, $\tilde h$ -- высота значительного волнения, полученная по форме
отраженного импульса.  

Сразу стоит оговориться, что в реальном эксперименте точность измерений высоты
значительного волнения определяется в первую очередь временн\'{ы}м разрешением
приемной антенны радиолокатора. Это означает, что при малой ширине импульса
(малой высоте значительного волнения) абсолютная ошибка может достигать 40 см, при высоте значительного волнения в 80 см. 

С ростом высоты волнения длительность импульса нелинейно увеличивается и уже
при высоте волнения в 1 м ошибка будет гораздо меньше и составлять 10 \%
от величины волнения.


\begin{figure}[ht]
    \centering
    \begin{subfigure}{0.49\linewidth}
        \centering
        \includegraphics[width=\linewidth,page=1]{fig/retracking/model}
    \end{subfigure}
    \begin{subfigure}{0.49\linewidth}
        \centering
        \includegraphics[width=\linewidth,page=3]{fig/retracking/model}
    \end{subfigure}
    \begin{subfigure}{0.49\linewidth}
        \centering
        \includegraphics[width=\linewidth,page=2]{fig/retracking/model}
    \end{subfigure}
    \caption{Отраженный импульс от моделируемых морских поверхностей при разных
    скоростях ветра (a) $U_{10}=5 ~\text{м}/\text{c}$, (b) $U_{10}=10
~\text{м}/\text{c}$, (c) $U_{10}=15 ~\text{м}/\text{c}$}
    \label{fig:model_pulse_retracking}
\end{figure}






\subsection{Восстановление параметров морской поверхности поверхности по данным
радиовысотомера}


Теперь применим алгоритм ретрекинга к реальным данным и восстановим высоту
значительного волнения по данным радиовысотомера космической миссии Jason-3. 

Данные находятся в открытом доступе на сайте
\href{https://data.nodc.noaa.gov/jason3/gdr}{NASA}. 
Данные представлены в необработанном виде, поэтому перед процедурой ретрекинга
необходимо усреднить полученные импульсы, а также учесть траекторию спутника и
не проводить вычисления для тех данных, где отсутствовала морская поверхность.


На рис. \ref{fig:impulse_jason} представлены обработанные формы импульсов. 
\begin{figure}[ht]
    \centering
    \begin{subfigure}{0.49\linewidth}
        \centering
        \includegraphics[width=\linewidth, page=4]{fig/retracking5}
    \end{subfigure}
    \begin{subfigure}{0.49\linewidth}
        \centering
        \includegraphics[width=\linewidth, page=5]{fig/retracking5}
    \end{subfigure}
    \begin{subfigure}{0.49\linewidth}
        \centering
        \includegraphics[width=\linewidth, page=6]{fig/retracking5}
    \end{subfigure}
    \caption{Форма отраженного импульса в зависимости от времени, полученного с
    радиовысотомера космической миссии Jason-3.}
    \label{fig:impulse_jason}
\end{figure}

По импульсам с рис. \ref{fig:impulse_jason} мы можем восстановить значение
высоты значительного волнения
\begin{equation}
    \begin{aligned}
        \label{eq:retracking_real}
        \tilde h = 0.98\\
        \tilde h = 0.76\\ 
        \tilde h = 1.09\\
    \end{aligned}
\end{equation}


По оценкам, полученным из \eqref{eq:retracking} и \eqref{eq:retracking_real}
можем сделать вывод о точности восстановленных данных: модель представленной
взволнованной морской поверхности хорошо симулирует реальные измерения, а
значит может использоваться в проведении экспериментов по отладки
радиоизмерительной аппаратуры и методов дистанционного зондирования. 
При этом в численном эксперимента мы можем с легкостью менять частоту
дискретизации приемной антенны, длину волны, высоту полета спутника и его угол
зондирования и таким образом находить оптимальную схему измерений для каждой
конкретной задачи.

\newpage
%!TEX root = ../diplom.tex
\section{Заключение}%
\label{sec:zakliuchenie}

В данной работе проводился численный эксперимент на взволнованной морской
поверхности, в этот эксперимент входили следующие этапы:
\begin{enumerate}
    \item моделирование морского волнения
    \item моделирование отраженного с орбитального радиолокатора импульса на модельной поверхности
    \item применение к отраженному импульсу алгоритма восстановления данных
    морского волнения и оценка точности используемого алгоритма
    восстановления
\end{enumerate}

Также был предложен метод по моделированию поверхности не гармоническими
функциями, как это делают обычно, а троихоидальными функциями. Это метод
позволяет учесть экспериментально известный факт, что у морской поверхности
площадь впадин немного превосходит площадь гребней. Подобный эффект важен в
дистанционном зондировании, поскольку его учет повышает точность позволяет
точнее установить расстояние от спутника до морской поверхности.  В разделе \ref{sub:cwm} получены основные статистические
свойства, а также представлена связь характеристик заостренной поверхности с
обычной.

На модельной поверхности был рассчитан отраженный с орбитального радиолокатора
импульс и оценена точность восстановления высоты значительного волнения 
при заданной длительности зондирующего импульса. 

В дальнейшем планируется продолжить работу над моделью заостренной поверхности
и включить в неё ещё эффект асимметрии переднего фронта волны относительно
заднего фронта.




\newpage



\bibliographystyle{gost2008}
\bibliography{sections/retracking, sections/water, sections/appendix}
\newpage

\appendix
\section{Техника безопасности}
Согласно ТОИ Р-45-084-01  при работе на персональном компьютере на человека
могут оказывать следующие опасные производственные факторы:
\begin{itemize}
    \item повышенный уровень электромагнитных излучений
    \item повышенный уровень статического электричества
    \item пониженная ионизация воздуха
    \item статические физические нагрузки
    \item перенапряжение зрительных анализаторов
\end{itemize}
Для предотвращения вредных производственных факторов выполняются следующие
меры, определяемые ГОСТ Р509232-96 \cite{gost}, ТОИ Р-45-084-01 \cite{toi} и
СанПин 2.2.2./2.4.1340-03 \cite{sanpin}
\begin{enumerate}
    \item используется антибликовое покрытие для монитора персонального
        компьютера для предотвращения световых бликов (п. 6.3. \cite{sanpin}).
    \item источники освещения в комнате расположены таким образов, чтобы их
        свет не создавал неравномерности распределения яркости на рабочем месте
        (пп. 6.5-6.9. \cite{sanpin}).
    \item рабочее место подобрано таким образом, чтобы не создавалось лишней
        статической физической нагрузки (п. 1.7. \cite{toi}).
    \item соблюдаются меры противопожарной безопасности (п. 3. \cite{toi})
    
    \item соблюдается удобство рабочего пространства (\cite{gost} пп.4.3-4.4,
        \cite{sanpin} п.9). 
    \item Для предотвращения перенапряжения глаз, продолжительность работы с компьютером
        без перерыва не превышает двух часов (\cite{toi} п.3.2) 
\end{enumerate}

\section{Программный код}
\label{sec:code}
Обращаем внимание на то, что весь код в листингах лишен обработчика исключений
и условных конструкций  для экономии места. Листинги приведены в ознакомительных
целях. Работоспособный программный код можно найти на
\href{https://github.com/kannab98/diplom/}{Github'е} автора.
\lstinputlisting[label={lst:spectrum}, caption={Вычисление спектра волнения как кусочно-аналитической функции}]{scripts/spectrum.py}
\lstinputlisting[label={lst:surface}, caption={}]{scripts/surface.py}
\lstinputlisting[label={lst:kernel}, caption={Реализации CUDA вычислений на языке Python}]{scripts/kernel.py}
\lstinputlisting[label={lst:pulse}, caption={}]{scripts/pulse.py}


\end{document}
