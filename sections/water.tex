%!TEX root = ../diplom.tex

\newcommand{\tK}{\widetilde K}
\section{Численное моделирование морского волнения}

Традиционный подход к моделированию морского волнения состоит в том, что спектр волнения представляется в виде суммы синусоид (гармоник), амплитуда которых вычисляется по спектру волнения [1а-3а]. Предполагается, что гармоники не взаимодействуют друг с другом, поэтому возвышения поверхности, орбитальные скорости, уклоны и другие характеристики волнения являются их суммой.

\subsection{Общие понятия}%
\label{sec:obshchie_poniatiia}
Определим ряд общих понятий, описывающих возвышение взволнованной морской поверхности в рамках теории случайных пространственно-временных полей. В этом случае поверхность представляется в виде суммы синусоидальных волн со случайными фазами 
\begin{equation}
    \label{eq:surface}
    \xi(\vec r,t) = \sum\limits_{n=-\infty}^{\infty} 
        A_n(\vec \kappa_n) e^{i(\omega_n t + \vec \kappa_n \vec r + \psi_n)},
\end{equation}
где $\psi_n$ -- случайная фаза,
равномерно распределенная в интервале от $0$ до  $2 \pi$, 
$A_n (\vec \kappa_n)$ -- комплексная амплитуда гармоники с волновым числом
$\vec \kappa_n$ и временной частотой  $\omega_n$, связанной с  $\vec \kappa_n$ известным
дисперсионным соотношением, полученным в \cite{cite:4}
\begin{equation}
    \omega(\kappa) = \sqrt{\kappa g + \alpha \kappa^3},
\end{equation}
где $g$ -- ускорение свободного падения,  $\alpha$ -- коэффициент, полученный
из экспериментов.


Корреляционную функцию $K_{\xi}(\vec r,t)$ поля  $\xi(\vec r, t) $ определим
стандартным образом \cite{cite:corr}.
 \begin{equation}
    \label{eq:corr}
    K_{\xi}\qty[\vec r_1, \vec r_2, t_1,t_2] = \mean{\xi(\vec r_1,t_1)\xi^*(\vec r_2,t_2)}
\end{equation}

Поле высот в нашей задаче считаем стационарным в широком смысле, то есть 
$K_{\xi}\qty[\vec r_{1},\vec r_{2},t_{1},t_{2}] = K_{\xi}\qty[\vec \rho = \vec
r_{2} - \vec r_1, \tau=t_{2}-t_{1}]$. Будем считать, гармоники
независимыми друг от друга, а значит перекрестные члены в уравнении
\eqref{eq:corr} занулятся.
Корреляционную функцию поверхности
\eqref{eq:surface} несложно посчитать
\begin{equation}
    \label{eq:surface_corr}
    K_{\xi}\qty[\vec \rho,\tau] = \sum\limits_{n=-\infty}^{\infty} 
    \frac{A_n^2}{2} 
    \exp{i \qty(\vec \kappa_n \vec \rho + \omega \tau)}
\end{equation}

Для решения задачи моделирования отраженного от морской поверхности импульса
достаточно рассматривать мгновенный снимок моделируемой поверхности, в момент
отражения
а значит можно положить $\tau = \const = 0$  и  тогда $K_\xi[\rho,\tau] = K_\xi [\rho]$.

В этом случае справедлива формула Винера-Хинчина \cite{cite:10}
\begin{equation}
    \label{eq:Viner-Hinchin}
    S_\xi(\vec k) \int\limits_{-\infty}^{\infty} K_\xi \qty[\vec \rho] \exp{- i
    \vec \kappa \vec \rho} \dd \rho. 
\end{equation}


Будем считать, что спектр морского волнения можно представить в виде функции с
разделяющимися переменными, где $S_{\xi}(\kappa)$ определяет зависимость
спектральной плотности мощности от волнового числа, а функция $\Phi(\kappa, \phi)$ -- 
описывает зависимость спектральной плотности мощности от азимутального угла для
выбранного волнового числа
\begin{equation}
    S_\xi(\vec \kappa) = S_\xi(\kappa) \Phi(\kappa, \phi),
\end{equation}
где $\kappa = \sqrt{\kappa_x^2 + \kappa_y^2}$,  $\phi = \arctg
\frac{\kappa_x}{\kappa_y}$. Для
удобства, угловое распределение нормируется так, чтобы
$$\int\limits_{-\infty}^{\infty} \Phi(\kappa,\phi) \dd
\phi = 1$$.


Для моделирования будет использоваться спектр волнения, который получен в
работе \cite{cite:6} и приведен в разделе отчета 2.1.

\subsection{Двумерная модель поверхностного волнения}%
\label{sec:dvumernaia_model_poverkhnostnogo_volneniia}

В соответствии с предыдущим разделом, для моделирования случайной поверхности
$\xi(\vec r,t)$ будем использовать её представление в виде суперпозиции
плоских волн с различными частотами и случайными фазами $\psi_{nm}$, бегущих
под разными азимутальными углами $\phi_m$ \cite{cite:11}:
\begin{figure}[H]
    \centering
    \includegraphics[scale=1]{fig/image65}
    \caption{Плотность вероятности случайной фазы $\phi$.}
    \label{fig:phase}
\end{figure}

\begin{equation}
    \label{eq:surface2d}
    \xi(\vec r,t) = \sum\limits_{n=1}^{N} \sum\limits_{m=1}^{M}
    A_n(\kappa_n) \cdot
    F_m(\kappa_n,\phi_m) \cos \qty(\omega_n t + \vec \kappa \vec r + \psi_{nm}),
\end{equation}
где $\psi_{nm}$ -- случайная фаза, равномерно распределенная в интервале от $0$
до $2 \pi$ (см. рис. \ref{fig:phase}). В соответствии с
центральной предельной теоремой \cite{cite:7}. 

Амплитуда $n$-ой гармоники $A_n$ есть
мощность на интервале $\Delta \kappa_n$, которая вычисляется по спектру моделируемой
поверхности $S_\xi(\kappa)$. Пользуясь формулами  \eqref{eq:surface_corr} и
\eqref{eq:Viner-Hinchin}) получим точное выражение для нахождения амплитуды
$n$-ой гармоники  $A_n$




\begin{gather}
    \frac{1}{(2 \pi)^2} = S_{\xi}(\vec \kappa) e^{i \vec \kappa \vec \rho} \dd \vec k = 
    \frac{1}{(2 \pi)^2} = 
        \int\limits_{-\infty}^{\infty}
        \int\limits_{- \pi}^{\pi} 
    S_\xi(\kappa) \Phi(\phi) \kappa e^{i \vec \kappa\vec \rho} \dd \kappa \dd \phi = \\
    \frac{1}{(2 \pi)^2} \int\limits_{-\infty}^{\infty} \kappa S_\xi
    (\kappa) e^{i \vec \kappa 
    \vec \rho} \dd \kappa = \sum\limits_{n=-\infty}^{\infty} \frac{(A_n(\vec
\kappa_n))^2}{2} e^{i \vec \kappa_n \vec \rho} 
\end{gather}

\begin{equation}
    \label{eq:Amplitude}
    A_n(\kappa_n) = \frac{1}{2 \pi} \sqrt{\int\limits_{\Delta \kappa_n} 2
        \kappa S_\xi(\kappa)
    \dd \kappa}
\end{equation}

При достаточно большом $n \to \infty$ ($\Delta \kappa_n \to 0$) можно интегрировать
прямоугольником
\begin{equation}
    A_n(\kappa_n) = \frac{1}{2 \pi} \sqrt{ 2 \kappa S_\xi(\kappa_n) \Delta
    \kappa_n}
\end{equation}
c погрешностью, пропорциональной $\Delta A_n \sim  \sqrt{\frac{\dd \kappa
    S_\xi(\kappa)}{\dd \kappa}
\Delta \kappa_n^2}$. 

Введем новое
обозначение для удобства $S(\kappa_n)\equiv \kappa_n S_\xi (\kappa_n)$.

Аналогично вычислению амплитуд, можно вычислить азимутальное распределение $F_m$  следующим образом:
\begin{equation}
    F_{nm}(\kappa_n,\phi_m) = \sqrt{\int\limits_{\Delta \phi_m}
    \Phi_{\xi}(\kappa_n,\phi) \dd \phi},
\end{equation}
где $\Delta \phi = \frac{2\pi}{M}$ -- шаг по азимутальному углу.
При малом шаге  $\Delta \phi$ c ошибкой, пропорциональной $\Delta F_{nm} \sim
\frac{\dd \Phi(\kappa_n,\phi)}{\dd \phi} \Delta \phi^2$, можно перейти к соотношению
\begin{equation}
    F_{nm} (\kappa_n,\phi_m) = \sqrt{\Phi_\xi(\kappa_n,\phi_m) \cdot \Delta \phi_m}
\end{equation}


\begin{figure}[ht]
        \centering
        \includegraphics[width=0.6\linewidth]{fig/full_spectrum1.png}
        \caption{Спектр высот $S(k)$ при меняющейся скорости ветра}
        \label{fig:spectrum_heights}
\end{figure}
\begin{figure}[h!]
    \begin{minipage}{0.49\linewidth}
        \centering
        \includegraphics[width=\linewidth]{fig/full_angles1.pdf}

        (a)
    \end{minipage}
    \begin{minipage}{0.49\linewidth}
        \centering
        \includegraphics[width=\linewidth]{fig/full_angles2.pdf}

        (b)
    \end{minipage}
    \caption{Функция углового распределения $\Phi_{\kappa}$ для разных значений
    отношения текущего волнового числа $\kappa$ к величине спектрального пика
$\kappa_m$}
    \label{fig:angles_distrib}
\end{figure}

Графики $S(\kappa)$ и  $\Phi_\xi(\kappa)$ для наглядности изображены на рис.
\ref{fig:spectrum_heights} и рис. \ref{fig:angles_distrib} соответственно.
Далее $\kappa_m$ будет называться ордината максимума функции  $S(\kappa)$. Стоит
заметить, что с ростом скорости ветра число используемых гармоник, необходимых
для получения одинакового качества моделирования,
возрастает. 
Это обусловлено тем, что растет интервал волновых чисел $\kappa$, на котором
определен спектр волнения. На рис.() показаны корреляционные функции волнения с
разным числом гармоник в случае равномерного выбора шага. 
На рис. \ref{fig:water} изображены поверхности,
построенные по формуле \eqref{eq:surface2d}.

\begin{figure}[h!]
    \begin{minipage}{0.49\linewidth}
        \centering
        \includegraphics[width=\linewidth]{img/water5}

        (a)
    \end{minipage}
    \begin{minipage}{0.49\linewidth}
        \centering
        \includegraphics[width=\linewidth]{img/water6}

        (b)
    \end{minipage}
    \begin{minipage}{0.49\linewidth}
        \centering
        \includegraphics[width=\linewidth]{img/water7}

        (c)
    \end{minipage}
    \begin{minipage}{0.49\linewidth}
        \centering
        \includegraphics[width=\linewidth]{img/water10}

        (d)
    \end{minipage}
    \caption{ Полутоновое изображение смоделированного поля высот для
        направления ветра $30^\circ$ и разных скоростей ветра
        (a) $U_{10} = 5 \text{м}/\text{c}$;
        (b) $U_{10} = 6 \text{м}/\text{c}$;
        (c) $U_{10} = 7 \text{м}/\text{c}$;
        (d) $U_{10} = 10 \text{м}/\text{c}$;
}
    \label{fig:water}
\end{figure}

Такой подход к моделированию морской поверхности является одним из самых простых и достаточно эффективным, но у него есть существенные недостатки.

Прежде всего, моделируемая поверхность получается симметричной, хотя реальная поверхность асимметрична: передний склон волны более крутой и короткий по сравнению с задним склоном.

Кроме того, площадь гребней меньше площади впадин для морского волнения, что также не находит отражения в свойствах моделируемой поверхности. Эти отличия 
модельной поверхности от морской поверхности не позволят смоделировать так называемые поправки на состояние морской поверхности [12a, 13a]. Как с этим бороться, будет обсуждаться в дальнейшем.

Для моделирования морской поверхности необходимо определиться с числом
гармоник. Надо отметить, что с ростом скорости ветра число используемых
гармоник, необходимых для получения одинакового качества моделирования, будет
возрастать. Это обусловлено тем, что увеличивается интервал волновых чисел
$\kappa$, на котором определен спектр волнения (см. рис.
\ref{fig:spectrum_heights}). 

Второй вопрос, который надо решить, состоит в том, как расположить гармоники по оси волновых чисел. 


Самый простой вариант – равномерный шаг, который можно определить следующим образом:
\begin{equation}
    \Delta \kappa = \frac{k_{cut}}{(N-1)}, \text{ где}
\end{equation}
$\kappa_{cut}$ -- граничное волновое число, $N$ -- число грамоник.

Однако если посмотреть на форму спектра, то задача усложняется тем, что спектр высот является узким и в основном сосредоточен вблизи пика (длинноволновая составляющая спектра волнения). Поэтому вариант «логарифмического» шага смотрится логичным и положения гармоник вычисляются следующим образом
\begin{equation}
    k_i = k_{i-1} \cdot \Delta  \kappa
\end{equation}

Критерием качества моделирования, а также оптимального выбора числа гармоник
была выбрана близость следующих корреляционных функций высот:
\begin{equation}
    \begin{gathered}
        \label{eq:KK}
        K[\rho] = \int\limits_{-\infty}^{\infty} S(\kappa) \cos(\kappa\rho) \dd \kappa\\
        \tK(\rho) = \sum\limits_{n=1}^{N} \frac{A_n^2}{2} \cos(\kappa_n \rho)
    \end{gathered}
\end{equation}



Сравнение корреляционной функции $\tK[\rho]$ полученной по модели, с
теоретической корреляционной функцией $K[\rho]$   позволит оценить качество модели.

На рис. 5 показаны корреляционные функции волнения с разным числом гармоник в
случае равномерного выбора шага. На рис. 6,  7 и 8  показаны корреляционные по
другим моделям.

Как было отмечено выше, с увеличением скорости ветра число гармоник, необходимых для получения одинакового качества моделирования, возрастает. На рис.(9) продемонстрирован этот эффект. 
Хорошо заметно, что с ростом скорости ветра $K[\rho]$   медленнее спадает к нулю,

\subsection{Метод <<отбеливания>> спектра для одной переменной}%
\label{sec:metod_otbelivaniia_spektra_}

Для оптимизации времени построения поверхности и уменьшения количества гармоник
без уменьшения качества моделирования, предлагается использовать следующий
метод.

Предположим, что при больших $\rho$ гармонические составляющие корреляционной
функции не зависят друг от друга и мы можем пренебречь их взаимной корреляцией.
Тогда мощность <<шума>> функции $\tK (\rho)$ определяется выражением
$\displaystyle \sigma^2_{\text{шум}} = \sum\limits_{n=1}^{N} \frac{1}{2}
\qty ( \frac{A^2_i}{2} )^2 \equiv \sum\limits_{n=1}^{N} \frac{b_i^2}{2}$.

В областях малых $\rho$, напротив, гармоники должны сильно взаимодействовать и
соответствующая мощность равна  $\tK^2(0) =
\qty(\sum\limits_{n=1}^{N} b_i)^2$ (см. \eqref{eq:KK} ).
Образуем величину
\begin{equation}
    \label{eq:Q}
    Q = \frac{\sigma_{\text{шум}}^2}{\tK^2(0)},
\end{equation}
которая характеризует относительную мощность шумов. Минимум этой величины
находится путём решения системы уравнений
\begin{gather}
    \frac{\partial Q}{\partial b_i} = 0, \text{ для } i=1,2,\dots, N. \\
    \frac{b_i \qty( \sum\limits_{n=1}^{N} b_i )^2 - 2 \sum\limits_{n=1}^{N} b_i
    \sum\limits_{n=1}^{N}  \frac{b_i^2}{2}}{\qty(\sum\limits_{n=1}^{N}
b_i)^4}=0
\end{gather}

Частным результатом её решения является $b_1 = b_2 = \dots = b_N$.

Спектр модельного поля при этом имеет близкий к белому вид, а выравнивание
амплитуд спектральных компонент поля $S(\kappa)$ сводится к разбиению области
определения спектра $[\kappa_{min},\kappa_{max}]$ на участки $\Delta
\kappa_i$, интегралы по
которым от функции  $S(\kappa)$ имеют одно и тоже значение $b_i = b_{0} =
\frac{\sigma^2}{N}$.

Заметим теперь, что рассуждая о способах разбиения интервала частот
$[\kappa_{min},
\kappa_{max}]$ на участки $\Delta \kappa_i$ мы оставляли нерешенным вопрос о выборе
расположения гармоник $\kappa_i$ внутри этих участков. Обычно  $\kappa_i$ ставится у
правой границы ячейки  $\Delta \kappa_i$. При этом, однако, оказывается, что
модельная корреляционная функция плохо совпадает с экспериментальной
корреляционной функцией в области малых  $\rho$. Для достижения лучшего
согласия следует потребовать сопряжения всех производных (от первого до $N$-го
порядка) функций $\tK[\rho]$ и  $K[\rho]$ при  $\rho=0$. 
Поскольку $K'[\rho] = \frac{\partial^2 K[\rho]}{\partial \rho^2}$, это условие эквивалентно
требованию сопряжения моментов спектра модельного и реального полей, которое
записывается в виде
 \begin{equation}
    \sum\limits_{n=1}^{N} b_i \kappa_i^{2p} 
    = \int\limits_{0}^{\infty} \kappa^{2p}S(\kappa) \dd \kappa, 
\end{equation}

Полученная система $N$ уравнений для $N$ неизвестных $\kappa_i$ не имеет общего
решения и потому может анализироваться лишь численно. Чтобы упростить решение
нашей задачи, потребуем облегченного, по сравнению с предыдущим, условия
сопряжения вторых моментов модельного и реального спектров высот
 \begin{equation}
    b_i \kappa_i^2 = \int\limits_{\Delta \kappa_i} \kappa^2 S(\kappa) \dd \kappa,
\end{equation}
где $b_i= A_i^2 / 2$

Из него непосредственно следует правило нахождения узлов $\kappa_i$ 
\begin{equation}
    \label{eq:ki}
    \boxed{
        \kappa_i = \sqrt{\frac{1}{b_0}} \int\limits_{\Delta k_i} \kappa^2
        S(\kappa) \dd \kappa. 
    }
\end{equation}

Формула \eqref{eq:ki} выведена для спектра высот поверхностного волнения. Когда
возникает необходимость моделирования уклонов, то необходима сделать замену
переменной $S(\kappa) \to k^2 S(\kappa)$, чтобы получить формулу для нахождения правила
расположения гармоник для уклонов

\begin{equation}
    \label{eq:ki_slopes}
    \boxed{
        \kappa_i = \sqrt{\frac{1}{b_0}} \int\limits_{\Delta \kappa_i}
        \kappa^4 S(\kappa) \dd \kappa. 
    }
\end{equation}

\begin{figure}[ht]
    \centering
    \includegraphics[width=0.6\linewidth]{fig/correlation_angles_wa.png}
    \caption{ Расположение гармоник по методу <<отбеливания>> спектра
    наклонов. }
    \label{fig:nodes}
\end{figure}

\begin{figure}[ht]
    \centering
    \includegraphics[width=0.6\linewidth]{fig/correlation_height_wa.png}
    \caption{ Расположение гармоник по методу <<отбеливания>> спектра
    высот. }
\end{figure}
Такой способ выбора расположения гармоник, как нетрудно убедиться, обеспечивает
сопряжение корреляционных функций реального и модельного полей по второй
производной в нуле, или, иначе говоря, равенство дисперсий кривизн этих
полей.

В результате метод <<отбеливания>> дает лучший результат из всех рассмотренных подходов. 

\begin{figure}[ht]
    \centering
    \begin{minipage}{0.49\linewidth}
        \centering
        \includegraphics[width=\linewidth]{fig/correlation_height_height2.png}


        (a)
    \end{minipage}
    \begin{minipage}{0.49\linewidth}
        \centering
        \includegraphics[width=\linewidth]{fig/correlation_angles_height2.png}



        (b)
    \end{minipage}

    \caption{ Корреляционные функции высот (a) и уклонов (b) при расположении гармоник
    по методу <<отбеливания>> спектра по формуле \eqref{eq:ki} }
    \label{fig:ki}
\end{figure}

\begin{figure}[h!]
    \centering
    \begin{minipage}{0.49\linewidth}
        \centering
        \includegraphics[width=\linewidth]{fig/correlation_height_slopes2.png}

        (a)
    \end{minipage}
    \begin{minipage}{0.49\linewidth}
        \centering
        \includegraphics[width=\linewidth]{fig/correlation_angles_slopes2.png}

        (b)


    \end{minipage}
    \caption{ Корреляционные функции высот (a) и уклонов (b) при расположении гармоник
    по методу <<отбеливания>> спектра по формуле \eqref{eq:ki_slopes} }
    \label{fig:ki_slopes}
\end{figure}

Из рис. \ref{fig:ki} и \ref{fig:ki_slopes} видно, что определение положения
гармоник по методу отбеливания является эффективным только для той переменной,
которая использовалась в процедуре отбеливания. Для другой переменной результат
получается не слишком хорошим, что свидетельствует о необходимости
использования другого подхода при необходимости одновременного моделирования
поля высот и поля уклонов.

\subsection{Метод <<отбеливания>> спектра для двух переменных}%
Для такой задачи необходима рассмотреть другую функцию
относительных шумов $Q$, например
\begin{equation}
    \label{eq:Q_modif}
    Q = \frac{\qty(\sigma^{\text{н}}_{\text{шум}})^2}{(\tK^\text{н}(0))^2}+
        \frac{\qty(\sigma^{\text{в}}_{\text{шум}})^2}{(\tK^\text{в}(0))^2},
\end{equation}
где индексы <<н>> и <<в>> соответствуют наклонам и высотам. Учитывая то, что
оба слагаемых в уравнении \eqref{eq:Q_modif} вещественны и положительны, то 
экстремум функции $Q$ можно найти, зная экстремум каждого слагаемого по отдельности. 


Тогда, гармоники, определяющие минимум первого слагаемого описываются
формулой \eqref{eq:ki}, а минимум второго -- формулой \eqref{eq:ki_slopes}.  

\begin{figure}[h!]
    \begin{subfigure}{0.49\linewidth}
        \includegraphics[width=\linewidth]{fig/fig1}
        \caption{}
    \end{subfigure}
    \begin{subfigure}{0.49\linewidth}
        \includegraphics[width=\linewidth]{fig/fig2}
        \caption{}
    \end{subfigure}
    \caption{Расположение гармоник для отбеливания (a) высот, (b) наклонов}
\end{figure}
\begin{figure}[h!]
    \centering
    \includegraphics[width=0.6\linewidth]{fig/fig3}
    \caption{Совмещенное расположение гармоник для отбеливания}
    \label{fig:}
\end{figure}

Таким образом, двумерный вариант метода отбеливания является эффективным
способом выбора расположения гармоник для численного моделирования морской
поверхности, задаваемой моделью спектра.




\subsection{Заостренная морская поверхность}
Как говорилось ранее, при моделировании морской поверхности синусоидами мы
получаем нулевое среднее значение высот, что не позволяет смоделировать
поправки на состояние морской поверхности. 

Ниже предлагается модель поверхности, с помощью которой эти поправки можно
учесть.


\subsubsection{Двумерный случай}%
\label{ssub:odnomernyi_sluchai}

Рассмотрим для начала задачу моделирования двумерной поверхности суммой гармоник с детерменированными амплитудами и случайными фазами
 \begin{equation}
     z = \sum\limits_{j=0}^{N} A_j \cos(k_j x + \psi_j)
 \end{equation}

Чтобы получить модель заостренной волны введем нелинейное преобразование координат
\begin{equation}
    \qty{x,z(x)} \longrightarrow \qty{x + D(x),z(x)},
\end{equation}
где $D(x)$ горизонтальное смещение
\begin{equation}
    D(x) =  \frac{i}{2\pi} \int\limits_{-\infty}^{\infty}   \hat z e^{ikx} \dd{k},
\end{equation}
а $S(k)$ -- прямое Фурье преобразование исходной поверхности
\begin{equation}
    S(k) = \int\limits_{-\infty}^{\infty} z(x) e^{-ikx} \dd x 
\end{equation}

В нашем случае, функция $D(x)$ примет вид: 
\begin{equation}
    \begin{cases}
    x = x_{0} \underbrace{
    - \sum\limits_{j=0}^{N} A_j \sin(k_j x_0 + \psi_j)
    }_{D(x)} \\
        z = \sum\limits_{j=0}^{N} \cos(k_j x_{0} + \psi_j)
    \end{cases}
\end{equation}

Иными словами мы будем моделировать волнение не суммой гармонических функций, а 
суммой трохоид. 

Для того, чтобы наше преобразование $D(x)$ имело физический смысл
необходимо, чтобы для каждой $j$-ой гармоники выполнялось соотношение 
\begin{equation}
    A_j k_j \ll 1 
\end{equation}

\paragraph{Статистические моменты}%

Запишем характеристическую функцию нового случайного процесса $z(x_0(x))$ по
определению
 \begin{equation}
    \label{eq:Phi1}
    \Phi(i\theta) = \mean{ e^{i \theta z(x_0(x))}}
\end{equation}
Поскольку процесс $z(x_0)$ стационарный, то от \eqref{eq:Phi1} можно перейти к
\begin{equation}
    \label{eq:Phi2}
    \Phi(i\theta) = \lim_{L \to \infty} \frac{1}{2L} \int\limits_{-L}^{L} e^{i \theta z(x_0)
    }\dd x = 
    \lim_{L \to \infty} \frac{1}{2L} \int\limits_{-L}^{L} e^{i \theta z(x_0)} \qty( 1 + D'(x_0) ) \dd x_0
\end{equation}

Поскольку $z(x_0)$ стационарный процесс, а  $D'(x_0)$ стационарен по нашему
определению, то  \eqref{eq:Phi2} преобразуется к виду
\begin{equation}
    \label{eq:Phi}
    \Phi(i\theta) = (1 - i \theta \sigma_1^2) 
    \exp(-\frac{1}{2} \theta^2 \sigma_0^2),
\end{equation}
где $\sigma^2_n = \int\limits_{-\infty}^{\infty}  k^n S(k) \dd k$ -- момент
$n$-го порядка спектра волнения.

Зная характеристическую функцию не сложно получить необходимые статистические
моменты дифференцируя \eqref{eq:Phi}
\begin{equation}
    m_n = i^{-n} \dv[n]{\Phi(i\theta)}{\theta} \eval_{\theta = 0}
\end{equation}

Следовательно, среднее и дисперсия случаного процесса $z(x_0)$ будут
равны
\begin{gather}
    \label{eq:}
    \mean{z} = - \sigma_1^2, \quad \mean{z^2} = \sigma_0^2 \\
    \mean{z^2} - \mean{z}^2 = \sigma_0^2 - \sigma_1^4
\end{gather}

Также не сложно получить связь наклонов в смещенных координатах $x$ с наклонами
в несмещенных координатах $x_0$ пользуясь определением наклонов
 \begin{equation}
    \label{eq:}
    z'(x) = \dv{z(x)}{x} = \frac{z'(x_0)}{1 + D'(x_0)}
\end{equation}



\subsubsection{Трехмерный случай}%

Для трехмерного случая Пирсон \cite{cite:pierson} предоставил решение
линеаризованных уравнений движения для невязкой жидкости в лагранжевых
координатахх. Он показал, что в глубокой воде положение частиц на свободной поверхности задается следующими параметрическими уравнениями
\begin{equation}
    \begin{cases}
        \label{eq:surface2dcwm}
        z(\vec r,t) = \sum\limits_{n=1}^{N} \sum\limits_{m=1}^{M}
        A_n(\kappa_n) \cdot
        F_m(\kappa_n,\phi_m) \cos \qty(\omega_n t + \vec \kappa \vec r_0 +
        \psi_{nm}),    \\
        x = x_0 - \sum\limits_{n=1}^{N} \sum\limits_{m=1}^{M}
        A_n(\kappa_n) \cdot
        F_m(\kappa_n,\phi_m) \cos\phi_m \sin\qty(\omega_n t + \vec \kappa \vec r_0 +
        \psi_{nm}),\\
        y = y_{0} - \sum\limits_{n=1}^{N} \sum\limits_{m=1}^{M}
        A_n(\kappa_n) \cdot
        F_m(\kappa_n,\phi_m) \sin \phi_m \sin \qty(\omega_n t + \vec \kappa \vec
        r_0 + \psi_{nm}),
    \end{cases}
\end{equation}
где $\vec \kappa$ -- двумерный волновой вектор,  
$\vec r_0 = (x_0, y_0)$, $\vec r = (x, y)$


\paragraph{Статистические моменты}
\label{par:statisticheskie_momenty}
В трехмерном случае вычисления аналогичны двумерному случаю, но более
громоздкие.  

Введем смешанный $\sigma_{\alpha \beta \gamma}^2$ и начальный $\sigma_n^2$ моменты спектра волнения
\begin{equation}
    \sigma^2_{\alpha \beta \gamma} \int\limits_{} \frac{k_x^\alpha
    k_y^\beta}{k^{\gamma}} S(\vec k) \dd \vec k,\quad
    \sigma_n^2 = \int\limits_{}^{} k^n S(\vec k) \dd \vec k 
\end{equation}
можно получить следующую характеристическую функцию для трехмерного волнения
\begin{equation}
    \Phi(\theta) = (1 - i \theta \sigma_1^2 + \theta^2 \Sigma_1)
    \exp(-\frac{1}{2} \theta^2 \sigma_0^2),
\end{equation}
где $\Sigma_1 = \sigma^4_{111} - \sigma_{201}^2 \sigma_{021}^2$.

Из этой характеристической функции можно получить необходимые моменты процесса
\begin{equation}
    \label{eq:}
    \mean{z} = - \sigma_1^2, \quad \mean{z^2} = \sigma_0^2 - 2 \Sigma_1
\end{equation}
 
На рис. \ref{fig:cwm} представлены срезы трехмерной морской поверхности 
для стандартного подхода и метода заостренной волны. 

На рис. \ref{fig:evolution} представлена эволюция во времени гребня волны для
двух методов. 
\begin{figure}[h!]
    \centering
    \includegraphics[width=0.8\linewidth]{fig/cwm}
    \caption{Срез поля высот морской поверхности для стандартного подхода и
    модели заостренной поверхности}
    \label{fig:cwm}
\end{figure}

\begin{figure}[h!]
    \centering
    \includegraphics[width=0.8\linewidth]{fig/evolution}
    \caption{Эволюция поверхности, построенной стандартным подходом в сравнении
    с моделью заостренной поверхности}
    \label{fig:evolution}
\end{figure}



На практике средний уровень морской поверхности не совпадает с тем, что может
определить альтиметр. Этот эффект возникает из-за того, что площадь впадин на
поверхности превышает площадь гребней, а значит во впадинах будет больше
отражающих зеркальных точек. 

Это приводит к увеличению длительности переднего фронта импульса, излучаемого
радиолокатором. Об этом речь пойдет позднее. 




 








%Рассчитаем спектр  поверхности, построенной по CWM.

 %Спектр поверхности в новых координатах 
 %\begin{equation}
     %\label{eq:S_cwm}
     %\widehat S(k) = \int\limits_{-\infty}^{\infty} z(x) \cdot  
     %e^{-ikx} \qty[ 1 + D'(x)]
     %\dd{x}  = S(k) + \int\limits_{-\infty}^{\infty} z(x) \cdot D'(x)  
     %e^{-ikx}\dd{x},
 %\end{equation}

 %где $S(k)$-- исходный спектр (например, JONSWAP).

 %Пусть спектр $S(k)$ состоит из всего одной гармоники. Тогда, мы можем его
 %записать в виде
 %\begin{equation}
     %S(k) = \pi A_{0}( \delta(k-k_{0}) + \delta(k+k_{0}))
 %\end{equation}

 %Для такого спектра функция горизонтального смещения и её производная равны:
 %\begin{equation}
     %D(x) = -A_{0} \sin(k_{0} x), \quad
     %D'(x) = -A_{0} k_{0} \cos(k_{0} x)
 %\end{equation}

 %Запишем прямое преобразование Фурье для функции $D'(x)$:
 %\newcommand{\D}{\mathfrak{D}}
  %\begin{equation}
     %\D(k) = \int\limits_{-\infty}^{\infty} D'(x) e^{-ikx} \dd x =
     %- \pi A_{0}k_{0}\qty[\delta(k-k_{0})+\delta(k+k_{0})]
 %\end{equation}

 %Перепишем \eqref{eq:S_cwm}, применяя теорему о свертке:
%\begin{equation}
    %\widehat S(k) = S(k) +
    %\frac{1}{2\pi} \int\limits_{-\infty}^{\infty} S(k-\xi) \D(\xi) \dd \xi 
%\end{equation}
%\begin{gather}
    %\widehat S(k) = S(k) + \frac{A_{0}}{2} \qty[\D(k-k_{0}) + \D(k+k_{0})] = \\
    %S(k) - \frac{\pi A_{0}^2}{2} k_{0} \qty[
    %\delta(k-2k_{0}) +
    %\delta(k+2k_{0}) +
    %\delta(k) +
    %\delta(k) 
    %] = \\
    %\pi A_{0}[\delta(k+k_{0})+ \delta(k-k_{0})] -
    %\frac{\pi A_{0}^2k_{0}}{2}\qty[ \delta(k+2k_{0}) + 2\delta(k) +
    %\delta(k-2k_{0})]
%\end{gather}
%Итак, для частного случае, когда моделируемая поверхность представляет всего
%одну гармонику $z(x) = A_{0}\cos(k_{0}x)$
%Мы получили модифицированный спектр:
%\begin{equation}
    %\widehat S = 
    %\pi A_{0}[\delta(k+k_{0})+ \delta(k-k_{0})] -
    %\frac{\pi A_{0}^2k_{0}}{2}\qty[ \delta(k+2k_{0}) + 2\delta(k) +
    %\delta(k-2k_{0})]
%\end{equation}

%Очевидно, что если поверхность будет представляться суммой гармоник
%$z(x) = \sum\limits_{n=0}^{N} A_n \cos{k_n x}$,
%спектр примет вид
%\begin{equation}
    %\widehat S = 
    %\underbrace{
    %\sum\limits_{n=0}^{N} \pi A_n \qty[\delta(k+k_n) + \delta(k-k_n) ]
%}_{S(k)}
%\underbrace{
%- \sum\limits_{n=0}^{N} \frac{\pi A_{n}^2 k_n}{2}
%\qty[ \delta(k+2k_{n}) 
    %+ 2\delta(k) 
%+ \delta(k-2k_{n})]
%}_{S_{CWM}(k)}
%\end{equation}

%Добавка к исходному спектру выглядит следующим образом
%\begin{gather}
    %\boxed{
    %S_{CWM}(k) = - \sum\limits_{n=0}^{N} \pi A_{n}^2 k_n \delta(k) - 
    %\sum\limits_{n=0}^{N} \frac{\pi A_n^2 k_n}{2} \qty[\delta(k+2k_n) +
    %\delta(k-2k_n)]
%}
%\end{gather}

%\begin{figure}[ht]
    %\begin{minipage}{0.49\linewidth}
        %\centering
        %\includegraphics[width=\linewidth]{fig/cwm_surface.pdf}

        %(a)
        %\caption{Заостренная синусоида (CWM) в сравнении обычной}
        %\label{fig:1}
    %\end{minipage}
    %\hfill
    %\begin{minipage}{0.49\linewidth}
        %\centering
        %\includegraphics[width=\linewidth]{fig/cwm_spectrum.pdf}

        %(b)
        %\caption{Спектр заостренной синусоиды}
        %\label{fig:2}
    %\end{minipage}
%\end{figure}
 %%\subsection{Модифицированный CWM}%
 
 %Для того, чтобы получить ассиметричную, заостренную поверхности введем другую
 %функцию горизонтального смещения
 %$D(x)$
 %\begin{equation}
     %D(x) = 
     %\begin{cases}
          %- A_{0} \sin(k_{0}x), &\text{ если }  0 \leq k_{0} x \leq  \pi \\
         %0, & \text{ если }  \pi < k_{0} x < 2 \pi
     %\end{cases}
 %\end{equation}
%И снова введем преобразование координат
%\begin{equation}
    %\qty{z(x),x} \to \qty{z(x), x + D(x)}
%\end{equation}
 %Спектр поверхности в новых координатах 
 %\begin{equation}
     %\widehat S(k) = \int\limits_{-\infty}^{\infty} z(x) \cdot  
     %e^{-ikx} \qty[ 1 + D'(x)]
     %\dd{x}  = S(k) + \int\limits_{-\infty}^{\infty} z(x) \cdot D'(x)  
     %e^{-ikx}\dd{x},
 %\end{equation}
 %где $S(k)$-- исходный спектр (например, JONSWAP).

%%\begin{figure}[h!]
    %%\begin{minipage}{0.45\linewidth}
        %%\centering
        %%\includegraphics[width=\linewidth]{example-image-a}
        %%\caption{График функции $D(x)$}
    %%\end{minipage}
    %%\hfill
    %%\begin{minipage}{0.45\linewidth}
        %%\centering
        %%\includegraphics[width=\linewidth]{example-image-a}
        %%\caption{График функции $D'(x)$}
    %%\end{minipage}
    %%\label{fig:}
%%\end{figure}

 %Задача сводится теперь к вычислению интеграла
 %\begin{equation}
     %I(k) = \int\limits_{-\infty}^{\infty} z(x) \cdot D'(x) e^{-ikx}\dd x 
 %\end{equation}



%Согласно теореме о свертке 
%\begin{equation}
    %\label{eq:I}
    %I(k) = \frac{1}{2\pi} \int\limits_{-\infty}^{\infty}  S(k-\xi) \D(\xi)\dd{\xi}, \text{ где}
%\end{equation}
%$\D(k) = \int\limits_{-\infty}^{\infty} D'(x) e^{-i k x}\dd{x} $,
%$S(k) = \int\limits_{-\infty}^{\infty} z(x) e^{-i k x}\dd{x} $ --
%обратное Фурье-преобразование функций $D'(x)$ и  $z(x)$ соответственно.
%\subsubsection{Нахождение спектральной плотности функции $D'(x)$}%
%Для простоты рассмотрим случай одной гармоники. 

%В этом случае $D(x)$ можно представить в виде модуляции гармонического сигнала
%прямоугольным сигналом.


%$D(x)$ представим в виде произведения двух функций 
%$D(x) = -A_{0}\sin(k_{0}x) \cdot  \Theta(k_{0} x)$,
%где $\Theta(k_{0}x)$ -- прямоугольный сигнал.

%Запишем прямое преобразование Фурье 
%\begin{gather}
    %\label{eq:D}
    %\D(k) = \int\limits_{-\infty}^{\infty} D'(x) e^{-ikx} \dd{x} = \\
    %- A_0 k_0\int\limits_{-\infty}^{\infty} 
     %\cos k_{0}x\cdot \Theta (k_{0} x) e^{-ikx}\dd x
     %- A_{0}
     %\underbrace{
     %\int\limits_{-\infty}^{\infty} 
    %\Theta'(k_{0} x) \sin(k_{0} x) e^{-ikx}\dd x
%}_{=~0} = \\
    %= \int\limits_{-\infty}^{\infty} 
    %\underbrace{- A_{0} k_{0} \cos k_{0}x\cdot}_{f(x)}
    %\underbrace{\Theta (k_{0} x) }_{g(x)}e^{-ikx}
    %\dd x = \int\limits_{-\infty}^{\infty} f(x) g(x) e^{-ikx} \dd x. 
%\end{gather}
%Осталось найти спектр функций $f(x)$ и  $g(x)$ и снова воспользоваться теоремой о свертке.

%\paragraph{Спектр функции $f(x)$.}%
%\begin{gather}
    %S_f(k) = - A_{0} k_{0}\int\limits_{-\infty}^{\infty} \frac{e^{+ik_{0}x} + e^{-ik_{0}x}}{2} e^{-ikx} \dd x = \\ 
    %- \frac{A_{0} k_{0}}{2}
    %\underbrace{ 
         %\int\limits_{-\infty}^{\infty} e^{-i(k-k_{0})x}\dd{x}  
    %}_{2 \pi \delta(k-k_{0})}
    %- \frac{A_{0} k_{0}}{2} 
    %\underbrace{
    %\int\limits_{-\infty}^{\infty} e^{-i(k-k_{0})x}\dd{x} 
    %}_{2 \pi \delta(k+k_{0})} = \\
    %-\frac{A_{0} k_{0}}{2} \cdot 2 \pi \qty{ \delta(k-k_{0}) + \delta(k+k_{0}) } = \\
    %\boxed{
    %- \pi A_{0} k_{0} \qty{ \delta(k-k_{0}) + \delta(k+k_{0})}
    %}
%\end{gather}

%\paragraph{Спектр функции $g(x)$.}%
%$X = \frac{2\pi}{k_{0}}$ -- период прямоугольного импульса, совпадающий с
%периодом синусоиды частотой 
%$k_0$. При этом $k_n = n k_0$
%\newcommand{\sinc}[1]{\textrm{sinc}\qty(#1)}
%\begin{gather}
    %C_n(k) = \frac{1}{X} \int\limits_{0}^{\frac{X}{2}} e^{-i k_n x} \dd{x} = 
    %\frac{1}{- i k_n X} e^{-i k_n x} \eval_{0}^{\frac{X}{2}} =
    %\frac{1}{- i k_n X} \qty( e^{-i \frac{k_n X}{2}} - 1) = \\
    %\frac{1}{- i k_n X} e^{-i \frac{k_n X}{4}}\qty( e^{-i \frac{k_n X}{4}} - e^{+i \frac{k_n X}{4}}) = \frac{e^{-i \frac{k_n X}{4}}}{2} \cdot \sinc{\frac{k_n X}{4}}
%\end{gather}

%\begin{gather}
    %\label{eq:G}
    %S_g(k) =\frac{e^{-i  \frac{\pi}{2 k_0}  k} }{2}\cdot
    %\sinc{\frac{\pi}{2k_{0}}k}\sum\limits_{n=-\infty}^{\infty}  2 \pi\delta(k - n
    %k_0)
%\end{gather}

%Вернемся к \eqref{eq:D} 
%\begin{equation}
    %\D(k) = \frac{1}{2 \pi} \int\limits_{-\infty}^{\infty} S_f(k-x) S_g(x)
    %\dd{x} = - \frac{A_{0} k_{0}}{2} \qty[S_g(k-k_{0}) + S_g(k+k_{0})].
%\end{equation}
%Переобозначим $\D_{\pm}(k) = - \frac{A_{0} k_{0}}{2} S_g(k \mp k_{0})$.
%Распишем теперь эту формулу, используя \eqref{eq:G}:
%\begin{equation}
    %\D_{\pm} = -\frac{A_{0}k_{0}}{2} \cdot \frac{e^{-i  \frac{\pi}{2 k_0}
    %(k\mp k_{0})} }{2}\cdot
    %\sinc{\frac{\pi}{2k_{0}}(k\mp k_{0})}\sum\limits_{n=-\infty}^{\infty} 2\pi  \delta(k - n
    %k_0)
%\end{equation}

%Вернемся теперь к уравнению \eqref{eq:I}. 
%Для него мы нашли $\D(k)$.
%в случае одной гармоники равен  $S(k) = \pi A_0 \qty{\delta(k-k_{0}) +
%\delta(k+k_{0})}$.
%Получаем


%\begin{gather}
    %I(k) = \frac{1}{2\pi} \int\limits_{-\infty}^{\infty}  S(k-\xi)
    %\D(\xi)\dd{\xi} = \\
    %\frac{1}{2\pi} \int\limits_{-\infty}^{\infty} \pi A_{0}\qty{
    %\delta(\xi- (k-k_{0}) ) + \delta(\xi - (k+k_{0})) } 
    %\cdot \qty{\D_{+}(\xi) + \D_{-}(\xi)} \dd \xi = \\
    %\frac{A_{0}}{2} (
    %\D_{+}(k-k_{0}) +
    %\D_{+}(k+k_{0}) +
    %\D_{-}(k+k_{0}) +
    %\D_{-}(k-k_{0}) 
    %) = \\
    %-\frac{A_{0}^2 k_{0}}{2} \qty[
    %S_g(k-2k_{0}) +
    %S_g(k+2k_{0}) +
    %2S_g(k)
    %]
%\end{gather}

%\begin{equation}
    %\boxed{
    %\widehat S(k) = S(k)  
    %-\frac{A_{0}^2 k_{0}}{2} \qty[
    %S_g(k-2k_{0}) +
    %S_g(k+2k_{0}) +
    %2S_g(k)
    %] }
%\end{equation}
%\begin{equation}
%\boxed{
    %S_g(k) =\frac{e^{-i  \frac{\pi}{2 k_0}  k} }{2}\cdot
    %\sinc{\frac{\pi}{2k_{0}}k}\sum\limits_{n=-\infty}^{\infty}  
    %2 \pi\delta(k - n k_0)
%}
%\end{equation}

%\appendix

%\newpage
%\section{Спектральное представление непериодических сигналов}%
%Пусть $U(t)$ одиночный импульс конечной длительности. Создадим периодическую
%последовательность с периодом $T$ и представим её комплексным рядом Фурье

%\begin{equation}
    %\label{eq:1}
    %U_{\text{периодич} } (t) = \sum\limits_{n=-\infty}^{\infty} C_n \exp{ i n \omega_0 t},
%\end{equation}
%где 
%\begin{equation}
    %\label{eq:2}
    %C_n =\frac{1}{T} \int\limits_{- \frac{T}{2}}^{\frac{T}{2}} U(t) \exp{-in\omega_{0}t } \dd t
%\end{equation}

%Для того, чтобы перейти к спектральному представлению единичного импульса,
%устремим $T \to  \infty$.

%Из \eqref{eq:2} видно, что при $T \to \infty$ получаем:
%\begin{enumerate}
    %\item Бесконечно-малые амплитудные коэффициенты $C_n$ (из-за наличия $T$ 
        %в знаменателе);
    %\item Частоты соседник гармоник $n \omega_{0}$ и $(n+1) \omega_{0}$ 
        %становятся  сколь угодно близкими (т.к. $\omega=\frac{2\pi}{T}$ ;
    %\item Число гармоник, входящих в ряд Фурье, становится бесконечно большим, т.к. при $T \to \infty$ основная частота $\omega_{0} = \frac{2 \pi}{T} \to 0$ ,
        %т.е. спектр становится сплошным.
%\end{enumerate}
%Подставим \eqref{eq:1}  в \eqref{eq:2}, получим: 
%\begin{equation}
    %\label{eq:3}
    %U(t) = \sum\limits_{n=-\infty}^{\infty} 
    %\qty( 
    %\int\limits_{- \frac{T}{2}}^{\frac{T}{2}} U(x) \exp(-in \omega_{0} t)
        %)  
        %\cdot \exp(in\omega_{0} t) \cdot \frac{\omega_{0}}{2 \pi},
%\end{equation}
%т.к. $T \to \infty $, то  $\omega_{0} = \frac{2\pi}{T} \to 0$, 
%а значит в \eqref{eq:3}  можно перейти от суммирования к интегрированию 
%$\omega_{0}$, $n \omega_{0} \to \omega$, 
%$\sum\limits_{n=-\infty}^{\infty} \to \int\limits_{-\infty}^{\infty}  $. 
%Таким образом, получаем двойной интеграл Фурье
%\begin{equation}
    %\label{eq:5}
    %U(t) = \frac{1}{2 \pi} \int\limits_{-\infty}^{\infty} e^{i \omega t} 
    %\qty[ 
    %\underbrace{ 
    %\int\limits_{-\infty}^{\infty} U(x) e^{- i \omega x} \dd x
%}_{S(\omega)}
        %] \dd \omega.
%\end{equation}

%\begin{equation}
    %\boxed{
    %S(\omega) = \int\limits_{-\infty}^{\infty} U(t) e^{-i\omega t} \dd{t} 
%}
%\end{equation}
%Функцию $S(\omega)$ здесь и далее будем называть 
%\textbf{прямым преобразованием Фурье} функции $U(t)$ или 
%\textbf{спектарльной плотностью сигнала} $U(t)$.

%С учетом обозначений, получим
%\begin{equation}
    %\label{eq:6}
    %\boxed{
    %U(t) = \frac{1}{2 \pi} \int\limits_{-\infty}^{\infty} 
            %S(\omega) e^{i \omega t} \dd \omega }
%\end{equation}
%\eqref{eq:6} есть \textbf{обратное преобразование Фурье}.

%Амплитудно-частотной характеристикой сигнала $U(t)$ будем называть 
%\begin{equation}
    %\abs{S(\omega)} = \sqrt{ \Re{S(\omega)}^2 + \Im{S(\omega)}^2} 
%\end{equation} 

%Фаза-частотной характеристикой сигнала $U(t)$ будем называть функцию
%\begin{equation}
    %\Theta(\omega) = \arctg{\frac{\Im{S(\omega)}}{\Re{S(\omega)}}}
%\end{equation}

%\section{Основные свойства преобразований Фурье}%
%\paragraph{Сложение сигналов}%
%Преобразование Фурье линейно.
%Если 
%\begin{equation}
    %U(t) = U_{1}(t) + U_{2}(t) + \dots + U_n(t),
%\end{equation}
%то 
%\begin{equation}
    %S(\omega) = S_{1}(\omega) + S_{2}(\omega) + \dots + S_n (\omega),
%\end{equation}


%\paragraph{Теорема запаздывания}%
%\begin{equation}
    %U_{2}(t) = U_{1}(t-t_{0})
%\end{equation}
%\begin{gather}
    %S_{2}(\omega) = \int U_1(t - t_{0}) e^{- i \omega t} \dd t =
    %\qty{\theta = t -t_{0}, \dd t = \dd \theta} = \\ 
    %\int\limits_{-\infty}^{\infty} 
    %U_1 (\theta) e^{- i \omega(\theta+t_{0})} \dd{\theta}  = e^{-i\omega t_{0}}
    %S_{1}(\omega);
%\end{gather}

%\begin{equation}
    %\boxed{
    %S_{2}(\omega) = e^{-i \omega t_{0}} S_{1}(\omega)}
%\end{equation}

%\paragraph{Изменение масштаба времени}%
%$U_2(t) = U_{1}(nt)$, $n>1$ -- сжатие сигнала, $n<1$ -- расширение сигнала.
%\begin{equation}
    %S_{2} (\omega) = \int\limits_{0}^{\frac{\tau}{n}} U_{2}(t) e^{-i\omega t}
    %= \int\limits_{0}^{\frac{\tau}{n}}  U_{1}(nt) e^{-i \omega t} \dd t.
%\end{equation}
%После замены переменных $nt = \theta, \dd t = \dd(\frac{\theta}{t})$
 %отсюда имеем
%\begin{equation}
    %S_{2}(\omega) = \frac{1}{n} \int\limits_{0}^{\frac{\tau}{n}}  
    %U_{1}(\theta) e^{-i \frac{\omega}{n} \theta} \dd \theta = \frac{1}{n}
    %S_1\qty(\frac{\omega}{n})
%\end{equation}
%\begin{equation}
    %\boxed{
        %S_{2}(\omega) = \frac{1}{n} S_{1}(\frac{\omega}{n})
    %}
%\end{equation}
%\paragraph{Произведение двух сигналов}%
%Рассмотрим составной сигнал $U(t) = f(t) \cdot g(t)$, где 
%$f(t) = \frac{1}{2 \pi} \int\limits_{-\infty}^{\infty} 
%F(\omega) e^{i \omega t} \dd \omega $, и 
%$g(t) = \frac{1}{2 \pi} \int\limits_{-\infty}^{\infty} 
%G(\omega) e^{i \omega t} \dd \omega $.  
%Найдём прямое преобразование Фурье:
%\begin{equation}
    %S(\omega) = \int\limits_{-\infty}^{\infty} 
    %f(t)\cdot g(t) e^{-i \omega t} \dd t = 
    %\frac{1}{(2 \pi)^2} \iiint F(x) G(y) \exp{- i(\omega - x - y)t} \dd{x} \dd{y} \dd{t} 
%\end{equation}
%Учтем, что 
%$\int\limits_{-\infty}^{\infty}   \exp{-i(\omega-x-y)t} \dd t = 
%2 \pi \delta(x+y - \omega) $
%\begin{equation}
    %\frac{1}{2 \pi}  \iint F(x) G(y) \delta(x+y-\omega) \dd{x} \dd{y}
%\end{equation}
%Применим фильтрующее свойство дельта-функции к функции $F(x)$ 
%\begin{equation}
    %S(\omega) = \frac{1}{2 \pi} \int\limits_{-\infty}^{\infty} 
    %F(\omega - y) G(y) \dd{y}
%\end{equation}
%\begin{equation}
    %\boxed{
        %S(\omega) = \frac{1}{2 \pi} \int\limits_{-\infty}^{\infty} 
        %G(x) F(\omega - x) \dd{x}  ~ 
    %} \text{ -- свертка спектров сомножителей.}
%\end{equation}

%уравнений
%\begin{thebibliography}{}
    %\bibitem{cite:1} \textit{М.С. Лонге-Хиггинс}, Статистический анализ случайно
    %движущейся поверхности // в книге Ветровые волны, Москва: Иностранная
    %литература, 1962, стр. 112-230.
    %\bibitem{cite:2} Статья по моделированию синусоидами 
    %\bibitem{cite:3} \textit{В. Караев, М. Каневский, Г. Баландина}, Численное
    %моделирование поверхностного волнения и дистанционное зондирование, 2000,
    %Препринт № 552, Нижний Новгород, изд. ИПФ РАН, 25 стр. 
    %\bibitem{cite:4} Дисперсионное уравнение
    %\bibitem{cite:5} \textit{В.И. Тихонов}, Статистическая радиотехника. // 2-е
    %изд., перераб. и доп. -- Москва: Радио и связь, 1982, стр. 119.
    %\bibitem{cite:6} Спектр Рябковой
    %\bibitem{cite:7} \textit{Гнеденко Б.В.}, Курс теории вероятностей: Учебник

    %\bibitem{cite:10} \textit{В.И. Тихонов}, Статистическая радиотехника. // 2-е
    %изд., перераб. и доп. -- Москва: Радио и связь, 1982, стр. 293.
    %для университетов. -- 6-е изд.  -- М.: Наука, 1988. -- \S 16 
    %стр. 400.
    %\bibitem{cite:12} \textit{Lee-Lueng Fu, Anby Cazenave}, Satellite altimetry
    %and earth sciences. A handbook of teckniques and applications, 2001,
    %Academic Press, 464 p.
    %\bibitem{cite:13} \textit{В. Пустовойтенко, А. Запевалов}, Оперативная
    %океанография: современное состояние, перспективы и проблемы спутниковой
    %альтиметрии, 2012, Севастополь, 218 с.
    %\end{thebibliography}

