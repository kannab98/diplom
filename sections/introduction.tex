
\section{Введение}%
\label{sec:vvedenie}

Моделирование морской поверхности является интересной научной задачей, которая
привлекает к себе внимание ученых на протяжении длительного времени.
Несмотря на значительный прогресс, остается много вопросов, которые требуют
дальнейших исследований. Кратко рассмотрим основные используемые подходы и
подробно обсудим способ моделирования, который будет использоваться в данной
работе.  Для описания поверхностного волнения применяют уравнения гидродинамики
и в общем виде задача пока не по силам современной вычислительной технике.
Благодаря упрощениям и предположениям задача становится <<счетной>>, но всё
равно требует много вычислительных ресурсов, поэтому этот подход используется для
решения научно-исследовательских задач. 

В настоящее время активно применяются радиофизические методы дистанционного
зондирования для
решения прикладных задач, например, для измерения скорости, ветра, высоты
значительного волнения, температуры воды, диагностики разливов нефти и др. 
Несмотря  на значительные успехи существующая измерительная аппаратура не
всегда позволяет получить достаточно полное представление о состоянии
приповерхностного слоя океана, поэтому постоянно разрабатываются новые
радиолокационные системы.

Разработка новой измерительной аппаратуры дистанционного зондирования для
космических носителей является сложной научно-технической задачей решение
которой требует проведения полного комплекса исследований, включающего
теоретический анализ, численное моделирование и эксперимент. В ходе решения
ищется оптимальная схема измерения, анализируются требования к измерительной
аппаратуре и оценивается эффективность алгоритмов обработки. Проведение
численного эксперимента является необходимым этапом, поскольку  позволяет максимально быстро проанализировать разные варианты и предложить схему измерения и состав измерительного комплекса для этапа экспериментальных работ.

В данной работе, кроме моделирования морской поверхности, будет проведен 
численный эксперимент с орбитальным радиолокатором в целях не только проверить
качество моделируемой поверхности, но и оценить точность алгоритмов
восстановления данных, которые обычно используются при обработки данных
радиовысотомера.  

Мы последовательно рассмотрим этапы такого численного эксперимента, начиная  с
моделирования морского волнения и заканчивая тестированием алгоритмов
восстановления параметров волнения по импульсу, отраженного от моделируемой поверхности.

