%!TEX root = ../diplom.tex
\section{Заключение}%
\label{sec:zakliuchenie}

В настоящее время роль численного моделирования при решении широкого круга
исследовательских и прикладных задач возрастает, что требует повышения
достоверности используемых моделей. В области дистанционного зондирования
морской поверхности орбитальными радиолокаторами применение численного
моделирования становится необходимым условием при разработке новой
измерительной аппаратуры.

Задачей данного исследования является проведение «численного» эксперимента,
включающего следующие этапы: 
\begin{enumerate}
    \item моделирование подстилающей поверхности (морское
волнение), на которой происходит рассеяние электромагнитных волн; 
    \item моделирование работы радиовысотомера и сравнение отраженного импульса с данными орбитальных радиовысотомеров; 
    \item реализация алгоритмов обработки и определения высоты значительнго
        волнения по форме отраженного импульса. Оценка эффективности алгоритма
        на примере модельных и реальных данных.
\end{enumerate}

 
Для моделивания морской поверхности был предложен и реализован в виде
программы, метод моделирования морской поверхности не гармоническими функциями,
а трохоидами. Благодаря этому достигается более реалистичная реализация морской
поверхности, учитывающая известный экспериментальный факт: площадь впадин
больше площади гребней, что приводит к ошибке при определении расстояния от
спутника до среднего уровня морской поверхности. Были получены основные
статистические характеристики <<заостренной>> поверхности и выполнено сравнение с
поверхностью, моделируемой синусоидами.

Был проведен численный эксперимент и для модельной поверхности был вычислен
отраженный импульс.  Для сравнения были скачаны измерения радиовысотомера
Jason.

Для обработки использовался алгоритм <<Ice>>, который был реализован в виде
отдельной подпрограммы. Алгоритм применялся для обработки данных Jason и
сравнение с результатами «штатной» обработки подтвердило его точность. Была
проведена обработка модельных данных и восстановлена высота значительного
волнения. Сравнение исходной и восстановленной высоты значительного волнения
подтвердило качество моделирования и перспективность применения численного
эксперимента для развития новой измерительной аппаратуры. 

В дальнейшем планируется продолжить работу над моделью заостренной поверхности
и включить в неё ещё эффект асимметрии переднего фронта волны относительно
заднего фронта.

