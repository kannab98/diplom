%!TEX root = ../diplom.tex
\section{Заключение}%
\label{sec:zakliuchenie}

В данной работе проводился численный эксперимент на взволнованной морской
поверхности, в этот эксперимент входили следующие этапы:
\begin{enumerate}
    \item моделирование морского волнения
    \item моделирование отраженного с орбитального радиолокатора импульса на модельной поверхности
    \item применение к отраженному импульсу алгоритма восстановления данных
    морского волнения и оценка точности используемого алгоритма
    восстановления
\end{enumerate}

Также был предложен метод по моделированию поверхности не гармоническими
функциями, как это делают обычно, а троихоидальными функциями. Это метод
позволяет учесть экспериментально известный факт, что у морской поверхности
площадь впадин немного превосходит площадь гребней. Подобный эффект важен в
дистанционном зондировании, поскольку его учет повышает точность позволяет
точнее установить расстояние от спутника до морской поверхности.  В разделе \ref{sub:cwm} получены основные статистические
свойства, а также представлена связь характеристик заостренной поверхности с
обычной.

На модельной поверхности был рассчитан отраженный с орбитального радиолокатора
импульс и оценена точность восстановления высоты значительного волнения 
при заданной длительности зондирующего импульса. 

В дальнейшем планируется продолжить работу над моделью заостренной поверхности
и включить в неё ещё эффект асимметрии переднего фронта волны относительно
заднего фронта.



