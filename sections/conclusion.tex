%!TEX root = ../diplom.tex
\section{Заключение}%
\label{sec:zakliuchenie}

В данной работе проводился численный эксперимент на взволнованной морской
поверхности, в этот эксперимент входили следующие этапы:
\begin{enumerate}
    \item моделирование морского волнения
    \item моделирование отраженного с орбитального радиолокатора импульса на модельной поверхности
    \item применение к отраженному импульсу алгоритма восстановления данных
    морского волнения и оценка точности используемого алгоритма
    восстановления
\end{enumerate}

Так же был предложен метод по моделированию поверхности не гармоническими
функциями, как это делают обычно, а троихоидальными функциями. Это метод
позволяет учесть экспериментально известный факт, что у морской поверхности
площадь впадин немного превосходит площадь гребней. Подобный эффект важен в
дистанционном зондировании, поскольку его учет повышает точность численного
эксперимента.  В разделе \ref{sub:cwm} получены основные статистические
свойства, а также представлена связь характеристик заостренной поверхности с
обычной.


На модельной поверхности был рассчитан отраженный с орбитального радиолокатора
импульс и оценена точность алгоритмов восстановления данных морского волнения. 
Благодаря модели заостренной поверхности удалось увеличить точность
моделирования и добиться относительной погрешности восстановления высоты
значительного волнения порядка 10-15 \% (см. предыдущий раздел), что делает
модельный эксперимент по точности сравнимым с реальными измерениями, а значит
делает пригодным для модельную поверхность для проведения численных
экспериментов для отладки измерительной аппаратуры.
